\lecture{Sept. 23}

\topic{Inequalities}
\begin{exmp}
Find all $x\in \mathbb{R}$ such that $$0<|x-2|\leq 4$$
\end{exmp} 

\begin{solution}
$[-2,6]$ with $x\neq 2$
\end{solution} 

\topic{Three Basic Inequalities}
\begin{enumerate}
\item $|x-a|<\delta$
\item $0<|x-a|<\delta$
\item $|x-a|\leq \delta$
\end{enumerate}

\begin{solution}

\begin{enumerate}
\item $(a-\delta, a+\delta) = \{x\in R \mid a-\delta < x < a+\delta \}$
\item $(a-\delta, a+\delta) \text{ with } x\neq a = \{x\in \mathbb{R} \mid a -\delta < x < a + \delta , x\neq a\}$
\item $[a-\delta, a+\delta] = \{x\in R \mid a-\delta \leq x \leq a+\delta \}$
\end{enumerate}
\end{solution}

\topic{Sequence}

\begin{defn}
A \textbf{sequence} is an infinite ordered list of real numbers.
\end{defn}

\begin{nota}
$\{1,2,3,4,\dots\}$ or  $\{1,\frac{1}{2},\frac{1}{3},\frac{1}{4},\dots\}$
\end{nota}


\begin{defn}
A \textbf{sequence} of real numbers is a function $a:\mathbb{N}\to\mathbb{R}$

The element $f(n)$ is called the n-th term of the sequence. We often denote this by $f_(n) = a_n$
\end{defn}


\begin{nota}
We can denote sequences in many ways
\begin{enumerate}
\item $f(n)=\frac{1}{n}$ for all $n\in \mathbb{N}$
\item Let $a_n = \frac{1}{n}$ 
\item $\{1,\frac{1}{2},\dots,\frac{1}{n},\dots\}$
\item $\{\frac{1}{n}\}$


\item Sometimes we define sequences recursively.

$a_1 = 1$ and $a_{n+1} = \sqrt{3+2a_n}$ for all $n\geq 1$.
\end{enumerate}
\end{nota}

\topic{Graphing Sequence}

\topic{Subsequence}
\begin{defn}
Let $\{a_n\}$ be a sequence, and let $\{n_k\}$ be a sequence of natural numbers with $n_1<n_2<n_3<\dots<n_k<n_{k+1}<\dots$.

The sequence $b_k = a_{n_k}\to \{b_k\}_{k=1}^\infty$ is called \textbf{subsequence} of $\{a_n\}$. We often write this as 
$$\{a_{n_1},a_{n_2},\dots,a_{n_k},\dots\}$$
\end{defn}

\topic{Important Subsequences}
Given $\{a_n\}$, let $n_0 \in \mathbb{N} \cup \{0\}$.
Define $$b_k = a_{n_0+k}$$
This sequence is called a tail of $\{a_n\}$

\topic{Limits of Sequences}
Consider $\{\frac{1}{n}\}=\{1,\frac{1}{2},\dots,\frac{1}{n},\dots\}$

\begin{note}
As $n$ gets larger and larger, the terms of the sequence $\{\frac{1}{n}\}$ get closer and closer to 0. We would like to say that the sequence $\{\frac{1}{n}\}$ converges to 0 and call 0 the limit of $\{\frac{1}{n}\}$.
\end{note}


\begin{defn}
\textbf{(Heuristic Definition of Convergence).} We say that a sequence $\{a_n\}$ has a limit $L$ if for every positive tolerance $\epsilon > 0$, the term $a_n$ will approximate $L$ with an error less than $\epsilon$ so long as the index $n$ is large enough.
\end{defn}





