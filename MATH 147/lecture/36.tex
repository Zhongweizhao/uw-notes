\lecture{Nov. 25}

\begin{thm}[Increasing Function Theorem]
Assume that $f(x)$ is continuous on $[a,b]$ and differentiable on $(a,b)$ with $f'(x) \geq 0$ ($f'(x) > 0$) for all $x\in (a,b)$, then $f(x)$ is (strictly) increasing on $(a,b)$.
\end{thm}


\begin{thm}[Comparison Theorem]
Assume that $f$ and $g$ are differentiable on $(a,b)$ and continuous on $[a,b]$. If $f(a) = g(a)$ and if $f'(x)<g'(x)$ for all $x \in (a,b)$, then $f(x) < g(x)$ for all $x \in (a,b]$. 
\end{thm}

\topic{Classifying Critical Points}

\begin{thm}[First Derivative Test]
Assume that $f'(c) = 0$

\begin{enumerate}
\item Assume that there exists an open interval $(a,b)$ containing $c$ with $f'(x) \geq 0$ for all $a<x<c$ and $f'(x) \leq 0$ for all $c<x<b$, then $x=c$ is a local maximum.

\item Assume that there exists an open interval $(a,b)$ containing $c$ with $f'(x) \leq 0$ for all $a<x<c$ and $f'(x) \geq 0$ for all $c<x<b$, then $x=c$ is a local minimum.
\end{enumerate}
\end{thm}

\begin{proof}
Let $a < x_0 < c$. Then Mean Value Theorem holds on $[x_0,c]$. There exists $d_1 \in (x_0 ,c)$ with \[\frac{f(x_0) - f(c)}{x_0 - c} = f'(d_1) \geq 0.\] Then $f(x_0) \leq f(c)$ since $x_0 - c < 0$. Similarly we prove the other parts of the theorem.
\end{proof}

\begin{thm}[Second Derivative Test]
Assume that $f'(c) = 0$ and that $f''(x)$ is continuous at $x=c$.
\begin{enumerate}
\item If $f''(c) > 0$, then $x = c$ is a local minimum
\item If $f''(c) < 0$, then $x= c$ is a local maximum
\end{enumerate}
\end{thm}

\begin{proof}
Assume that $f'(c) = 0$ and that $f''(x)$ is continuous at $x=c$.
\begin{enumerate}
\item Assume $f''(x) > 0$. Since $f''(x)$ is continuous at $x=c$, there is an open interval $(c-\delta, c+\delta)$ on which $f''(x) > 0$. Hence $f'(x)$ is strictly increasing on $(c-\delta,c+\delta)$. But $f'(c) <0$, then $f'(x) < 0$ on $(c-\delta , c)$ and $f'(c) > 0$ on $(c,c+\delta)$. Then we apply the First Derivative Test.
\end{enumerate}
\end{proof}

\begin{defn}[Concavity]
We say that a function $f(x)$ which is continuous on an interval $I$ is concave up on $I$ if for every $a< b$, $a,b \in I$, we have
\[h(x) = f(a) + \frac{f(b) - f(a)}{b-a} - f(x) \geq 0 \text{ on }(a,b)\]
We say that a function $f(x)$ is concave down on $I$ if for every $a< b$, $a,b \in I$, we have
\[h(x) = f(a) + \frac{f(b) - f(a)}{b-a} - f(x) \leq 0 \text{ on }(a,b)\]
\end{defn}

\begin{thm}[Concavity Theorem]\leavevmode

\begin{enumerate}
\item Assume that $f''(x) > 0$ for all $x\in I$ then $f(x)$ is concave up on $I$.
\item Assume that $f''(x) < 0$ for all $x\in I$ then $f(x)$ is concave down on $I$.
\end{enumerate}
\end{thm}