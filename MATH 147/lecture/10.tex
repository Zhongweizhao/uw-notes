\lecture{Sept. 30}

\topic{Series}

\begin{defn}
A series $\sum_{n=1}^\infty a_n$ is \textbf{positive} is for all $n\in \mathbb{N}$, if $S_k=\sum_{n=1}^k a_n$, then $S_{k+1} - S_k = a_{k+1}\geq 0$
\end{defn}

\begin{exmp}
\textbf{Harmonic Series} Does $\displaystyle\sum_{n=1}^\infty \frac{1}{n}$ converge?
\end{exmp}

Let $\displaystyle S_k = \sum_{n=1}^k \frac{1}{n}$,

\begin{align*}
    S_1 &= 1 = \frac{2}{2}\\
    S_2 &= 1 + \frac{1}{2} =\frac{3}{2}\\
    S_4 &= 1+ \frac{1}{2} +\frac{1}{3}+\frac{1}{4}>1+\frac{1}{2}+\frac{1}{4}+\frac{1}{4} = \frac{4}{2}\\
    S_8 &= 1 + \frac{1}{2} +\frac{1}{3} +\frac{1}{4} +\frac{1}{5} +\frac{1}{6} +\frac{1}{7} +\frac{1}{8} \\
        &> 1 + \frac{1}{2} +\frac{1}{4} +\frac{1}{4} +\frac{1}{8} +\frac{1}{8} +\frac{1}{8} +\frac{1}{8} = \frac{5}{2}\\
    & \vdots \\
    S_{2^k} &> \frac{2+k}{2}
\end{align*}

Since $\displaystyle\{\frac{2+k}{2}\}$ is not bounded, $\{S_k\}$ is not bounded.

\begin{exmp}
$\displaystyle\sum_{n=2}^\infty \frac{1}{n^2-n}$
\end{exmp}

\begin{note}
\begin{align*}
    \frac{1}{n^2-n} & = \frac{1}{n(n-1)}\\
    & = \frac{1}{n-1} - \frac{1}{n}
\end{align*}
\end{note}

\begin{solution}

\begin{align*}
    S_1 &= 1 - \frac{1}{2} = 1 - \frac{1}{2}\\
    S_2 &= 1 - \frac{1}{2} + \frac{1}{2} - \frac{1}{3} = 1 - \frac{1}{3}\\
     S_3 &= 1 - \frac{1}{2} + \frac{1}{2} - \frac{1}{3} + \frac{1}{3} - \frac{1}{4} = 1 - \frac{1}{4}\\
     &\vdots \\
     S_k & = 1 - \frac{1}{k}
\end{align*}


As $k\to \infty$, $\displaystyle\sum_{n=2}^\infty \frac{1}{n^2-n} = 1$
\end{solution}


\begin{exmp}
$\displaystyle\sum_{n=1}^\infty \frac{1}{n^2}$
\end{exmp}

\begin{note}
For $n\geq 2$, $$\frac{1}{n^2}< \frac{1}{n^2-n}$$
\end{note}

\begin{align*}
    T_k = \sum_{n=1}^k \frac{1}{n^2} &= 1+ \frac{1}{2^2}+ \frac{1}{3^2}+\dots + \frac{1}{k^2}\\
    &< 1 + \frac{1}{2^2-2}+ \frac{1}{3^2-2}+\dots + \frac{1}{k^2-k}\\
    &< 1 + 1\\
    &= 2
\end{align*}

Since $T_k\leq 2$ for all k, $\{T_k\}$ is bounded and by the Monotone Convergence Theorem is convergent with $\displaystyle 1\leq\sum_{n=1}^\infty \frac{1}{n^2}\leq 2 $.

In fact, $\displaystyle\sum_{n=1}^\infty \frac{1}{n^2} = \frac{\pi^2}{6}$


\begin{exmp}
Consider $\displaystyle\sum_{n=1}^\infty \frac{1}{n!}$, does this converge?
\end{exmp}

Note that $\displaystyle\frac{1}{n!}<\frac{1}{2^n}$ for $n\geq k$.

In fact, $\displaystyle\sum_{n=1}^\infty \frac{1}{n!} = e$


\begin{note}
$$\sum_{n=1}^\infty \frac{(-1)^{n+1}}{n} =1 -  \frac{1}{2}+\frac{1}{3}-\frac{1}{4}+\dots$$
\end{note} 


\topic{Arithmetic Rules for Sequences}

\begin{ques}
Assume $a_n\to 3$, $b_n\to 7$.

What can you say about

\begin{enumerate}
\item[1)] $\{4a_n\}$
\item[2)] $\{a_nb_n\}$
\item[3)] $\{a_n+b_n\}$
\item[4)] $\displaystyle\{\frac{a_n}{b_n}\}$
\end{enumerate}

\end{ques}

\begin{thm}
\textbf{Arithmetic Rules for Sequences} Let $\{a_n\}$, $\{b_n\}$ be such that $\lim_{n\to\infty}a_n = L$, $\lim_{n\to\infty}b_n = M$.

Then
\begin{enumerate}
\item[1)] $\lim_{n\to\infty} ca_n = cL$ for all $c\in \mathbb{R}$
\item[2)] $\lim_{n\to\infty} a_n+b_n = L + M$
\item[3)] $\lim_{n\to\infty} a_nb_n = LM$
\item[4)] $\displaystyle\lim_{n\to \infty} \frac{1}{a_n} = \frac{1}{L}$ if $L\neq 0$
\item[5)] $\displaystyle \lim_{n\to\infty} \frac{a_n}{b_n} = \frac{L}{M}$ if $M\neq 0$
\end{enumerate}
\end{thm}

\begin{proof}
\begin{enumerate}
\item[1)] If $c=0$ then $ca_n = 0$ for all n. Hence $\lim_{n\to\infty} ca_n =\lim_{n\to\infty} 0 = 0L = cL$
Suppose $c\neq 0$, Let $\epsilon > 0$. We want $N$ so that if $n\geq N$, $\displaystyle |ca_n - cL| < \epsilon \Leftrightarrow |a_n-L|<\frac{\epsilon}{|c|}$

Choose $N_0$ such that if $n\geq N_0$ we have $\displaystyle |a_n-L|<\frac{\epsilon}{|c|}$

If $n\geq N_0$, $$|ca_n - cL|\leq |a_n - L||c|<\frac{\epsilon}{|c|}|c|=\epsilon$$

\end{enumerate}
\end{proof}




