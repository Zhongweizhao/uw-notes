\lecture{Oct. 24}

Midterm: 7:00-8:45

RCH 307 - A-J

RCH 306 - K-O

DWE 3522 - P-W

DWE 3522A - X-Z

Woman in Pure Math/Math Finance Lunch

Tuesday 12:30-1:20 MC5417

\begin{defn}
We say that $L$ is the limit of $f(x)$ from above (from the right) if for every $\epsilon > 0$ there exists $\delta > 0$ such that if $0<x-a<\delta$, then $\abs{f(x) - L}<\epsilon$. We write \[\lim_{x\to a^+} f(x) = L\]
We say that $L$ is the limit of $f(x)$ from below (from the left) if for every $\epsilon > 0$ there exists $\delta > 0$ such that if $-\delta<x-a<0$, then $\abs{f(x) - L}<\epsilon$. We write \[\lim_{x\to a^-} f(x) = L\]
\end{defn}

\begin{note}
Both the Arithmetic Rules and Sequential Characterization hold for one-sided limits. As does the Squeeze Theorem.
\[lim_{x\to a^+} f(x) = L \text{ iff whenever } \{x_n\} \text{ is such that } x_n\to a, a<x_n \text{ we have } \lim_{x\to\infty} f(x_n) = L\]
\end{note}

\begin{thm}
The following are equivaent
\begin{enumerate}
\item \(\lim_{x\to a} f(x) = L\)
\item \(\lim_{x\to a^-} f(x) = L\) and \(\lim_{x\to a^+} f(x) = L\)
\end{enumerate}
\end{thm}

\begin{proof}
\begin{enumerate}
\item Assume that \(\lim_{x\to a} f(x) = L\), Let \(\epsilon > 0\), then there exists \(\delta>0\) such that if $0 < \abs{x-a} < \delta$ then $\abs{f(x)- L} < \epsilon$.  Hence if $0<x-a < \delta$ then $\abs{f(x) - L} < \epsilon$ and if $0<a-x < \delta$ then $\abs{f(x) - L} < \epsilon$. Thus \(\lim_{x\to a^-} f(x) = L\) and \(\lim_{x\to a^+} f(x) = L\).
\item Conversely, assume that \(\lim_{x\to a^-} f(x) = L\) and \(\lim_{x\to a^+} f(x) = L\). Let $\epsilon > 0$. We can find $\delta_1 > 0$ such that if $0 < x-a < \delta_1$ then $\abs{f(x) - L} < \delta$ and $\delta_2 > 0$ such that if $0 < a-x < \delta_1$ then $\abs{f(x) - L} < \delta$. Let $\delta = min\{\delta_1,\delta_2\}$, hence if $0 < \abs{x-a} < \delta$ then $\abs{f(x) - L} < \epsilon$.
\end{enumerate}
\end{proof}

\begin{exmp}
\[lim_{x\to 0} \frac{\abs{x}}{x}\]
\end{exmp}

\begin{exmp}
\[lim_{x\to 0^+} \sqrt{x} = 0\]
\end{exmp}

\begin{defn}
A function $f(x)$ is \textbf{even} if $f(x) = f(-x)$ for all $x\in\mathbb{R}$ (graph is symmetric about $x=0$)
\end{defn}

\begin{note}
If $f(x)$ is even, (assume these limits exist)
\[\lim_{x\to a^+} f(x) = \lim_{x\to -a^-} f(x)\]
\[\lim_{x\to a^-} f(x) = \lim_{x\to -a^+} f(x)\]
In particular, $\lim_{x\to 0} f(x)$ exists iff $\lim_{x\to 0^+} f(x)$ exists iff $\lim_{x\to 0^-} f(x)$ exists.
\end{note}

\begin{defn}
A function $f(x)$ is \textbf{odd} if $f(x) = -f(-x)$ for all $x\in\mathbb{R}$ (graph is symmetric about $(0,0)$)
\end{defn}
\begin{note}
If $f(x)$ is odd, (assume these limits exist)
\[\lim_{x\to a^+} f(x) = -\lim_{x\to -a^-} f(x)\]
\[\lim_{x\to a^-} f(x) = -\lim_{x\to -a^+} f(x)\]
$\lim_{x\to 0} f(x)$ exists iff $\lim_{x\to 0^+} f(x) = 0$ or $\lim_{x\to 0^-} f(x) = 0$
\end{note}

\begin{exmp}
$\lim_{x\to 0} \sin x$ and $\lim_{x\to 0} \cos x$
\end{exmp}

\begin{exmp}
\[\lim_{x\to 0} \frac{\sin x}{x} = 1\]
\end{exmp}