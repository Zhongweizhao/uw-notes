\lecture{Nov. 30}

\begin{exmp}
\[\lim_{x\to 0} \frac{e^x-1}{x}\]
\end{exmp}


\begin{exmp}
\[\lim_{x\to 0} \frac{e^x-\cos x}{x}\]
\end{exmp}

\begin{exmp}
\[\lim_{x\to 0} \frac{x}{e^x}\]
\end{exmp}

\begin{exmp}
\[\lim_{x\to 0^+} x\ln x\]
\end{exmp}

\begin{exmp}
\[\lim_{x\to 0^+} x^x\]
\end{exmp}

\begin{exmp}
\[\lim_{x\to 0^+} x^{\sin x}\]
\end{exmp}

\begin{defn}
Recall: If $f(x)$ is differentiable at $x = a$, then $L_a^f(x) = f(a) + f'(a)(x-a)$ is the unique degree 1 (or less) polynomial with 1) $L_a^f(a) = f(a)$ 2) $L_a^{f'(a)}=f'(a)$
\end{defn}

Question: Assume that $f''(a)$ exists, does there exist a polynomial $p(x) = a_0 +a_1(x-a)+a_2(x-a)^2$
with $p(a) = f(a)$, $p'(a) = f'(a)$, and $p''(a) = f''(a)$?

Note: 
\begin{align*}
    &p(a) = a_0, p'(x) = a_1 + 2a_2(x-a)\\
    &p'(a) = a_1 \Rightarrow Le + a_1 = f'(a)\\
    &p''(x) = 2a_2 \Rightarrow 2a_2 = f''(a)\\
    &p(x) = f(a) + f'(a)(x-a) + f''(a)(x-a)^2\\
    &p(x) = \frac{f(a)}{0!} + \frac{f'(a)}{1!}(x-a) + \frac{f''(a)}{2!}(x-a)^2\\
\end{align*}

Question: 
Assume that $f^{(n)}_{(a)}$ exists. Then is there a polynomial of the form $P_{n,a}(x) = a_0 + a_1(x-a) +...+ a_n(x-a)^n$ where $P^{(k)}_{n,0}(a) = f_{(a)}^{(k)}$ for $k$ = $0,1,2,...,n$?\\

The answer is YES, in fact (we define $0! = 1$):
\begin{align*}
    &P_{n,a}(x) = \frac{f(a)}{0!} + \frac{f'(a)}{1!}(x-a) + \frac{f''(a)}{2!}(x-a)^2 + ... + \frac{f^{(n)}(a)}{n!}(x-a)^n\\
    &P_{n,a}(x) = \sum_{k=1}^n\frac{f^{(k)}(a)}{k!}(x-a)^k
\end{align*}



\begin{defn}[n-th degree Taylor Polynomial]
Given a function $f(x)$ with $f^{(n)}(a)$. We define the n-th degree Taylor Polynomial for $f(x)$ centered at $x=a$ to be
\[P_{n,a}(x) = \frac{f(a)}{0!} + \frac{f'(a)}{1!}(x-a) + \frac{f''(a)}{2!}(x-a)^2 + ... + \frac{f^{(n)}(a)}{n!}(x-a)^n\]
\[P_{n,a}(x) = \sum_{k=0}^n \frac{f^{(k)}(a)}{k!}(x-a)^k\]
\end{defn}

Observation:
\begin{enumerate}
    \item $P_{0,a}(x) = f(a)$
    \item $P_{1,a}(x) = L_a^f(x)$
\end{enumerate}

\begin{exmp}
    $f(x) = e^x, a = 0$
    \begin{align*}
        &f'(x) = e^x = f''(x) = f'''(x) = ... = f^{(n)}(x)\\
        &f^{k}(0)=e^0=1
    \end{align*}
    \begin{align*}
        &P_{0,0}(x) = e^0 = 1\\
        &P_{1,0}(x) = f(0) + f'(0)x = 1 + x\\
        &P_{2,0}(x) = 1+x+ \frac{f''(a)}{2!}x^2 = 1 + x + \frac{x^2}{2!}\\
        &P_{n,0}(x) = 1+x+\frac{x^2}{2!}+\frac{x^3}{3!}+\frac{x^n}{n!}
    \end{align*}
\end{exmp}

Strategy:
If $x \approxeq a$, $f(x) \approxeq P_{n,0}(x)$ (Hope: As n increases the approximation is more accurate)

Question: Can we quantify the error? i.e. How big is $f'(x)-P_{n,0}(x)$?


\begin{defn}
\[R_{n,a}(x) = f(x) - P_{n,a}(x)\] is the error in using the Taylor Polynomial to approximate $f(x)$ near $x=a$.
\end{defn}

\begin{thm}[Taylor's Theorem]
    Assume that $f^{(n+1)}(x)$ exists in an open interval $I$ containing $x = a$ For each $x\in I (x \neq a)$ there exist $c_x$ strictly between x and a, such that
    \[R_{n,a}(x)=f(x)-P_{n,a}(x) = \frac{f^{(x+1)}(x)}{(n+1)!}(x-a)^{x+1}\]
\end{thm}

\begin{exmp}
    \begin{align*}
        &n = 1\\
        &R_{1,a}(x) = f(x) - P_{1,a}(x)\\
        &=f(x) - L_a^f(x)\\
        &=\frac{f''(x)}{2!}(x-a)^2
    \end{align*}
\end{exmp}