\lecture{Sept. 12}

\topic{Mathematical tools}
\LaTeX

MikTex, Winshell

\topic{Basics on Sets and Functions}
\begin{defn}\topic{Basic Sets}
\begin{itemize}
\item $\mathbb{N} = $ Natural numbers  $=\{1,2,3,\dots\}$
\item $\mathbb{Z} = $ Integers $ = \{\dots,-3,-2,-1,0,1,2,3,\dots \}$
\item $\mathbb{Q} = \{\frac{m}{n} \mid n\in \mathbb{N}, m\in \mathbb{Z},gcd(n,|m|)=1\}$
\item $\mathbb{R} = $ Real Numbers
\item $\mathbb{R}\backslash \mathbb{Q} = \{x\in \mathbb{R}  \mid  x\text{ is not in } \mathbb{Q}\}$
\end{itemize}
\end{defn}

\begin{nota}
\leavevmode

$S \subset X \to S \text{ is a subset of } X $

If $S, T \subset X$ then $S\cup T = \{x\in X\mid x\in S \text{ or } x\in T\}$

If $S, T \subset X$ then $S\cap T = \{x\in X\mid x\in S \text{ and } x\in T\}$

Given a collection $\{A_\alpha\}_{\alpha \in I}$ of subsets of $X$

$$\bigcup_{\alpha \in I} A_\alpha = \{x\in X \mid x\in A_\alpha \text{ for some } \alpha\in I\}$$
$$\bigcap_{\alpha \in I} A_\alpha = \{x\in X \mid x\in A_\alpha \text{ for all } \alpha\in I\}$$

$\emptyset=$ empty set, $\emptyset \subset X$

What if $I = \emptyset$, what is $\displaystyle \bigcup_{\alpha \in \emptyset}A_\alpha $

Define
$$\bigcup_{\alpha \in \emptyset} A_\alpha = \emptyset$$

Then 
$$\bigcap_{\alpha \in \emptyset} A_\alpha = ??$$

Given $S,T\subset X$ we define

$$S\backslash T = \{x\in X \mid x\in S , x \text{ does not belong to } T\}$$

We denote $X\backslash T$ by $T^c$ = compliment of $T$ in $X$ =   $ \{x\in X \mid x \text{ does not belong to } T\}$

\end{nota}

\begin{note}

$$(S\cup T)^c = S^c\cap T^c$$

\end{note}


\topic{De Morgans Law}

\begin{thm}
$$(\bigcup_{\alpha \in I}A_\alpha)^c = \bigcap_{\alpha \in I} A_\alpha^c$$
\end{thm}
\begin{proof}
\begin{align*}
x\in (\bigcup_{\alpha \in I}A_\alpha)^c & \iff x \text{ is not a member of } \bigcup_{\alpha \in I}A_\alpha\\
&\iff x \text{ is not in } A_\alpha \quad \forall \alpha \in I\\
&\iff x\in A_\alpha^c \quad \forall \alpha \in I\\
&\iff x\in \bigcap_{\alpha \in I} A_\alpha^c
\end{align*}
\end{proof}

\begin{note}
From this we really should have 

\begin{align*}
    \bigcap_{\alpha \in \emptyset} A_\alpha &= (\bigcup_{\alpha \in \emptyset} A_\alpha^c)^c \\ 
    & = \emptyset ^ c \\
    & = X
\end{align*}
\end{note}


\topic{Power Set}
\begin{defn}
Given $X$, the Power Set of $X$ is the set of all subset of $X$
\end{defn}

\begin{nota}
\begin{align*}
    P(X) & = \text{ power set of } X\\
    & = \{S \mid S\subset X\}\\
\end{align*}
\end{nota}

\begin{note}

We can observe that
    $$\emptyset ,X \in P(X)$$
    
\end{note}

