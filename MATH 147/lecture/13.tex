\lecture{Oct. 6}

\topic{Squeeze Theorem}

\begin{exmp}
Find $$\lim_{n\to\infty} \frac{cos(n)}{n}$$
\end{exmp}
\textbf{Observation: } 
\[\abs{cos(n)} \leq 1\]
\[\frac{-1}{n} \leq \frac{cos(n)}{n}\leq \frac{1}{n}\]

%graph y=1/n y=-1/n

\begin{thm}\textbf{Squeeze Theorem}
If $\{a_n\}$, $\{b_n\}$, $\{c_n\}$ are such that $a_n\leq b_n\leq c_n$ with $\lim_{n\to\infty} a_n = L = \lim_{n\to\infty} c_n$, then $\lim_{n\to\infty} b_n = L$
\end{thm}

\begin{proof}
Let $\epsilon > 0$, then exists $N_0\in\mathbb{N}$ so that if $n\geq N_0$ then $a_n \in (L-\epsilon,L+\epsilon)$ and $c_n \in (L-\epsilon,L+\epsilon)$

If $n\geq N_0$, \[L-\epsilon < a_n \leq b_n \leq c_n < L+\epsilon\]
\[\abs{b_n - L} < \epsilon\]
\end{proof}

\begin{solution}

We know that \[\frac{-1}{n} \leq \frac{cos(n)}{n}\leq \frac{1}{n}\]

since $\abs{cos(n)} \leq 1$

Since $\lim_{n\to\infty} -\frac{1}{n} = 0 = \lim_{n\to \infty} \frac{1}{n}$

Then 
\[\lim_{n\to\infty} \frac{cos(n)}{n} = 0\]


\end{solution}

\begin{exmp}
\[
\lim_{n\to\infty} (1+\frac{1}{n})^n = e
\]
\end{exmp}

\begin{note}
If $\{a_n\}$ is bounded, then 
\[
\lim_{n\to\infty} \frac{a_n}{n} = 0
\]
\end{note}

\topic{Bolzano-Weierstrass Theorem}

\begin{note}
We know that convergent sequences are bounded. But bounded sequences do not have to converge.

Does every bounded sequences have a convergent sub-sequence?



\textbf{Strategy} Bounded + monotonic $\Rightarrow$ convergent

Does every sequence have a monotonic sub-sequence

\end{note}

\begin{defn}
Given $\{a_n\}$ we call an index $n_0$ a \textbf{peak point} for $\{a_n\}$ if $a_n < a_{n_0}$ for all $n\geq n_0$
\end{defn}

\begin{lem} \textbf{Peak Point Lemma}
Every sequence $\{a_n\}$ has a monotonic sub-sequence.
\end{lem}

\begin{proof}
Let $P = \{n \in\mathbb{N} \mid \textit{n is a peak point of } \{a_n\}\}$ 

Case 1. P is infinite.

Let $n_1 = $ least element of $P$

Let $n_2 = $ least element of $P \\ \{n_1\}$

$\cdots$

This gives us a sequence recursively
\[n_1<n_2<\dots < n_k < \dots \in P\]

Since these are peak points, 
\[a_{n_k} > a_{n_{k+1}}\]

Thus $\{a_{n_k}\}$ is decreasing.

Case 2. Let $n_1$ be the least index that is not a peak point. Since $n_1$ is not a peak point, we can choose $n_2>n_1$ so that $$a_{n_1}\leq a_{n_2}$$

Since $n_2$ is not a peak point, then we can choose $n_3>n_2$ so that $$a_{n_2}\leq a_{n_3}$$

We can proceed recursively, to find that 
\[
n_1 < n_2 < \dots < n_k < \dots
\]

Where $a_{n_k}\leq a_{n_{k+1}}$

Thus $\{a_{n_k}\}$ is non-decreasing.

In either case we have a monotonic sub-sequence.
\end{proof}


\begin{thm} \textbf{Bolzano-Weierstrass Theorem}
Every bounded sequences has a convergent sub-sequence.
\end{thm}

\begin{proof}
Give $\{a_n\}$, by the Peak Point Lemma $\{a_n\}$ has a monotinic subsequence $\{a_{n_k}\}$, which is also bounded. By the MCT, $\{a_{n_k}\}$ is convergent.
\end{proof}

\begin{note}
BWT is equivalent to MCT which is equivalent to the LUBP.
\end{note}



