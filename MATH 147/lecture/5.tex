\lecture{Sept. 21}

\begin{enumerate}
\item[1)] No office hours this afternoon
\item[2)] WA1\(\to\) Due 2:30 PM Monday, Sept. 26. Submit in dropbox outside Math Tutorial Center.
\end{enumerate}

\topic{Least Upper Bound Property}
If \(S\subset \mathbb{R}\) is non-empty and bounded above, then \(S\) has a least upper bound.

\topic{Archimedean Property I}
\begin{thm}
\(\mathbb{N}\) is not bounded above.
\end{thm}

\begin{proof}
Suppose that \(\mathbb{N}\) was bounded above. Then \(\mathbb{N}\) has a least upper bound \(\alpha\). 

Note that \(\alpha - \frac{1}{2} < \alpha \). Hence \(\alpha -\frac{1}{2}\) is not an upper bound for \(\mathbb{N}\). Then there exists \(n\in \mathbb{N}\) with \(\alpha - \frac{1}{2}<n\leq \alpha\). But then \(n+1\in \mathbb{N}\) and \(n+1>\alpha\) which is impossible.

Therefore \(\mathbb{N}\) must not be bounded above.

\end{proof}


\begin{note}
 Let \(S\neq \emptyset\subset\mathbb{R}\) be bounded above.
Let \(\alpha = lub(S)\). if \(\epsilon > 0\) then there exist \(x_0\in S\) with \(\alpha - \epsilon < x_0 \leq \alpha\).
\end{note}


\topic{Archimedean Property II}
\begin{cor}
Let \(\epsilon > 0 \), Then there exists \(n\in \mathbb{N}\) such that \[0<\frac{1}{n}<\epsilon\]
\end{cor}

\begin{proof}
Take \(\alpha=\frac{1}{\epsilon}\) in Archimedean Property I.
\end{proof}

\topic{Density of \(\mathbb{R}\)}
\begin{defn}
A subset \(S\subset\mathbb{R}\) is said to be dense if for every \(\epsilon > 0\) and \(x\in \mathbb{R}\), \[S\cap (x-\epsilon,x+\epsilon)\neq \emptyset\]

or equivalently if 
\(\displaystyle S\cap (a,b)\neq \emptyset\) for all \(a<b \text{ in } \mathbb{R}\)
\end{defn}

\begin{prop}
\(\mathbb{Q}\) and \(\mathbb{R}\backslash \mathbb{Q}=\mathbb{Q}^c\) are dense in \(\mathbb{R}\)
\end{prop}

\topic{Absolute Values}
\begin{defn}
\begin{equation*}
f(x)=\begin{cases}
	x & x\geq 0\\
    -x & x<0
\end{cases}
\end{equation*}
\end{defn}

\begin{exmp}
\[g(x) = \frac{|x|}{x}\]
Domain \(= \{x\in \mathbb{R} \mid x\neq 0\}\)

\begin{equation*}
g(x) = \begin{cases}
	1 & x> 0 \\
    -1 & x<0
\end{cases}
\end{equation*}
\end{exmp}

\topic{Geometric Interpretation of \(|x|\)}
\begin{itemize}
\item \(|x|\) represents the distance from \(x\) to 0.
\item \(|x-a|\) represents the distance from \(x\) to \(a\).
\end{itemize}

\begin{note}
Distance between \((0,0)\) and \((x,y)\)
\[\sqrt{x^2+y^2}\]
\end{note}
 
\topic{Properties of \(|x|\)}
\begin{enumerate}
\item[1)] \(|x|\geq 0\) and \(|x| = 0 \iff x=0\)
\item[2)] \(|ax|=|a||x|\) for all \(a\in \mathbb{R}, x\in \mathbb{R}\)
\item[3)] Triangle Inequality
	\[|x-z|+|z-y|\geq |x-y|\]
\end{enumerate}


\begin{thm}
\textbf{Triangle Inequality} If \(x,y,z\in \mathbb{R}\), then	\[|x-z|+|z-y|\geq |x-y|\]
\end{thm}

\begin{proof}
Use Geometric Interpretation. % add graph
\end{proof}

\begin{thm}\textbf{Variants I}
For all \(x,y\in \mathbb{R}\),
\[|x+y|\leq |x|+|y|\]
\end{thm}

\begin{thm}\textbf{Variants II}
For all \(x,y\in \mathbb{R}\),
\[||x|-|y||\leq |x-y|\]
\end{thm}












