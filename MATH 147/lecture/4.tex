\lecture{Sept. 19}

\topic{Least Upper Bound Property}

\topic{Upper Bound}

\begin{thm}
Let $S\subset \mathbb{R}$ then $\alpha \in \mathbb{R}$ is an upper bound for $S$ if $x\leq \alpha$ for all $x\in S$. We say that $S$ is bounded above if $S$ has an upper bound.

We say that $\beta$ is a lower bound for $S$ if $\beta \leq x$ for all $x\in S$. We say that $S$ is bounded below if $S$ has a lower bound.

We say that $S$ is bounded if it is bounded above and below.
\end{thm}

\begin{exmp} Let $S=\{x_1,x_2,\dots ,x_n\}$ be finite.

By relabeling, if necessary we can assume that \[x_1<x_2<\dots <x_n\]

Then $\beta = x_1$, $\beta$ is a lower bound and $\alpha = x_n$ is an upper bound.

\end{exmp}

\begin{thm}
Every finite set is bounded.
\end{thm}

\begin{exmp}
Let $S=[0,1)=\{x\in \mathbb{R} \mid 0\leq x < 1\}$ (finite interval)

5 is an upper bound. -1 is a lower bound.

1 is also an upper bound. Moreover if $\gamma$ is any upper bound of $S$, then $1 \leq \gamma$
\end{exmp}

\topic{Least Upper Bound}


\begin{thm}
We say that $\alpha$ is the least upper bound of a set $S\subset \mathbb{R}$ if 
\begin{enumerate}
\item[1)] $\alpha$ is an uppper bound of $S$
\item[2)] if $\gamma$ is an upper bound of $S$, then $\alpha \leq \gamma$
\end{enumerate}
 
We write $$\alpha = lub(S)$$

(Sometimes $\alpha$ os called the supremum of $S$ and is denoted by $\alpha = sup(S)$)
\end{thm}

Back to the example $S=[0,1)$. 0 is a lower bound and if $\gamma$ is any lower bound, then $\gamma \leq 0$



\topic{Greatest Upper Bound}
\begin{thm}
We say that $\beta$ is the greatest lower bound of a set $S\subset \mathbb{R}$ if 
\begin{enumerate}
\item[1)] $\beta$ is an lower bound of $S$
\item[2)] if $\gamma$ is an lower bound of $S$, then $\gamma \leq \beta$
\end{enumerate}
 
We write $$\beta = glb(S)$$

(Sometimes $\beta$ os called the infimum of $S$ and is denoted by $\beta = inf(S)$)
\end{thm}

\begin{exmp}
if $S = [0,1)$, $lub(S)=1$, $glb(S)=0$.
\end{exmp}


\begin{note}
Is $\emptyset$ bounded (above or below)?

Note: 6 is an upper bound for $\emptyset$. If not, there exists an element in $\emptyset$ that is greater than 6. Similarly, 6 is a lower bound.

In fact, if $\gamma \in \mathbb{R}$ then $\gamma$ is both an upper and a lower bound of $\emptyset$. $\emptyset$ is a bounded set.
\end{note}

\begin{exmp}
Let $S=\{x\in \mathbb{Q} \mid x^2<2 \} \subset \mathbb{R}$

$\sqrt{2}$ is an upper bound and $-\sqrt{2}$ is a lower bound. And $lub(S) = \sqrt{2}$, $glb(S) = -\sqrt{2}$
\end{exmp}

\begin{exmp}

Let $S=\{x\in \mathbb{Q} \mid x^2<2 \} \subset \mathbb{Q}$

S does not have a least upper bound or a greatest lower bound.

\end{exmp}

\begin{ques}
If $S\subset R$ is bounded above, does it always have a least upper bound?
\end{ques} 

\topic{Least Upper Bound Property}


\begin{thm}
If $S\subset R$ is non-empty and bounded above, then $S$ has a least upper bound.
\end{thm}

\topic{Observation}
\begin{enumerate}
\item[1)] $\emptyset$ does not have a $lub$
\item[2)] If we only have rational numbers in the world, then $S=\{x\mid x^2<2\}$ does not have a lub. In other words, Least Upper Bound Property fails for $\mathbb{Q}$
\end{enumerate}


\begin{ques}
is $\mathbb{N}$ bounded?
\end{ques}

\begin{enumerate}
\item[1)] $\mathbb{N}$ is bounded below, $glb(S) = 1$
\end{enumerate}





