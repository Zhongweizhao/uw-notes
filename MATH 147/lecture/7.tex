\lecture{Sept. 26}

Writing Assignment 2 is due Friday Oct 14th.

\topic{Convergence of Sequences}
\begin{defn} \textbf{Heuristic definition I }
We say that a sequence $\{a_n\}$ converges to a limit $L$ if as n gets larger and larger the $a_n$s get closer and closer to $L$.
\end{defn}
\begin{defn} \textbf{Heuristic definition II }
We say that a sequence $\{a_n\}$ converges to a limit $L$ if for every positive tolerance $\epsilon > 0$, we have that the terms in $\{a_n\}$ approximate $L$ with an error at most $\epsilon$, provided that $n$ is large enough.
\end{defn}

\begin{defn}
\textbf{Convergence of a Sequence} We say that $\{a_n\}$ converges to a limit $L$ if for every $\epsilon > 0$, there exists a cutoff $N_0\in \mathbb{N}$ such that if $n\geq N_0$, then $|a_n-L|<\epsilon$

If no such $L$ exists, we say that $\{a_n\}$ \textbf{diverges}.
\end{defn}



\begin{exmp}
Consider $\{(-1)^{n+1}\} = \{1,-1,1,-1,\dots\} $. Does this have a limit?
\end{exmp} 

\begin{proof}
Let $\epsilon = 1$. Suppose $L = \lim_{n\to \infty} a_n$. Let $N_0$ be such that if $n\geq N_0$, then $|a-L|<1$

Let $n_1 \geq N_0$ with $n_0$ even. Then
\begin{align*}
    |-1-L| &= |a_n - L| < 1\\
\to L &\in (-2,0)
\end{align*}

Let $n_1 \geq N_0$ with $n_0$ odd. Then 
\begin{align*}
    |1-L| &= |a_n - L| < 1\\
\to L &\in (0,2)
\end{align*}

So \begin{equation*}
    L \in (-2,0) \cap (0,2)
\end{equation*}
which is impossible.

Hence $\{a_n\}$ diverges.
\end{proof}


\begin{note}
Suppose that $\lim_{n\to \infty} a_n = L$. Let $\epsilon > 0$. What can we say about the terms in $\{a_n\}$ that are in $(L-\epsilon ,L+\epsilon)$?

For some $N_0$, if $n\geq N_0$, then $a_n \in (L-\epsilon ,L+\epsilon)$. ie) $(L-\epsilon ,L+\epsilon)$ contains a tail of the sequence.

\end{note}

\begin{prop}
Let $\{a_n\}$ be a sequence. Then the following are equivalent.
\begin{enumerate}
\item $L = \lim_{n\to \infty} a_n$
\item for every $ \epsilon > 0 ,\  (L-\epsilon ,L+\epsilon) $ contains a tail of $\{a_n\}$
\item for every $\epsilon > 0 ,\ (L-\epsilon ,L+\epsilon) $ contains all but finitely many $a_n$
\item for open interval $(a,b)$ with $L\in (a,b)$, we have $(a,b)$ contains a tail of $\{a_n\}$
\item for open interval $(a,b)$ with $L\in (a,b)$, the interval $(a,b)$ contains all but finitely many $a_n$
\end{enumerate}
\end{prop}

\begin{ques}
Can $\{a_n\}$ have more than 1 limit?
\end{ques}

\begin{thm} \textbf{Uniqueness of Limit}
Suppose that $\lim_{n\to \infty} a_n = L$ and $\lim_{n\to \infty} a_n = M$, then $L = M$
\end{thm}
\begin{proof}
Assume that $L<M$. Let $\epsilon = \frac{M-L}{2}$.

We can choose $N_1$ large enough so that if $n \geq N_1$, $a_n \in (L-\epsilon ,L+\epsilon)$

We can also choose $N_2$ large enough so that if $n \geq N_2$, $a_n \in (M-\epsilon ,M+\epsilon)$

Let $N_0 = max \{N_1, N_2\}$. Choose $n \geq N_0$. Then $a_n \in (L-\epsilon ,L+\epsilon) \cap (M-\epsilon ,M+\epsilon)$

But $(L-\epsilon ,L+\epsilon) \cap (M-\epsilon ,M+\epsilon) = \emptyset$

\end{proof}












