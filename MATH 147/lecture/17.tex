\lecture{Oct. 20}


\topic{Seq Characteriation of Limits}
\begin{thm}
Let $f(x)$ be defined in an open interval I containing $a$, except possibly at $x=a$. Then the following are equivalent.
\begin{enumerate}
\item \(\displaystyle\lim_{x\to a} f(x) = L\)
\item Whenever $\{x_n\}$ is such that $x_n\to a$ ($x_n\neq a$) we have $f(x)\to L$
\end{enumerate}
\end{thm}

\begin{exmp}
$\displaystyle \lim_{x\to 0} \sin (\frac{1}{x})$ does not exist.
\end{exmp}

\begin{exmp}
\[g(x) = \begin{cases}
    x\sin(\frac{1}{x}) & \text{ if } x\neq 0\\
    0 & \text{ if } x= 0
\end{cases}\]

$\displaystyle\lim_{x\to 0} g(x) = 0$. In other words, the limit exists (by using squeeze theorem.)
\end{exmp}

\begin{exmp}
\[\lim_{x\to 0} x^2\sin (\frac{1}{x})= 0\]
\end{exmp}

\begin{exmp}
\[f(x) = \begin{cases}
0 & \text{ if } x\in \mathbb{R}\backslash\mathbb{Q} \\
1 & \text{ if } x= 0 \\
\frac{1}{m} & \text{ if } x = \frac{k}{m} \in \mathbb{Q} \text{ with } gcd(k,m) = 1
\end{cases}\]
Suppose $\lim_{x\to a} f(x)$ exists. Then the limit is $0$ (because for every irrational sequence that approaches $a$, all element in the irrational sequence is $0$.)
\end{exmp}

\begin{defn}
We say that $L$ is the limit of $f(x)$ from above (from the right) if for every $\epsilon > 0$ there exists $\delta > 0$ such that if $0<x-a<\delta$, then $\abs{f(x) - L}<\epsilon$. We write \[\lim_{x\to a^+} f(x) = L\]
We say that $L$ is the limit of $f(x)$ from below (from the left) if for every $\epsilon > 0$ there exists $\delta > 0$ such that if $-\delta<x-a<0$, then $\abs{f(x) - L}<\epsilon$. We write \[\lim_{x\to a^-} f(x) = L\]
\end{defn}