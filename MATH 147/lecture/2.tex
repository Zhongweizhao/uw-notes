\lecture{Sept. 14}

\topic{New Section}
12:30-1:20 CPH 3604\\
Tutorial Moved to DC 1350 Th 4:30-5:20

\topic{Greek Letters}
\begin{itemize}
    \item $\alpha$ - alpha
    \item $\beta$ - beta
    \item $\delta$ - delta
    \item $\epsilon$ - epsilon
    \item $\gamma$ - gamma
\end{itemize}

\topic{Properties of $\mathbb{N}$}
$$\mathbb{N} = \{1,2,3,4,\dots\}$$



\topic{Mathematical Induction}
\newtheorem{mathinduct}{Axiom}

\begin{axiom}
Assume $S\in \mathbb{N}$ such that 
\begin{enumerate}
\item $1\in S$
\item If $k\in S$, then $k+1 \in S$
\end{enumerate}
Then $S=\mathbb{N}$
\end{axiom}

\topic{Proof by Induction}
\begin{enumerate}
\item Establish for each $n\in \mathbb{N}$ a statement $P(n)$ to be proved.

Example. Let $P(n)$ be the statement that $\sum_{i=1}^n i = \frac{n(n+1)}{2}$, show this is true for all $n\in \mathbb{N}$.


    Let $S = \{n\in \mathbb{N}  \mid P(n) \text{ is true}\}$, show $S=\mathbb{N}$

\item Base Case: show that $P(1)$ is true. ie): $1\in S$
\item Inductive Step: Assume that $P(k)$ is true for some $k$ (Inductive Hypothesis). Use this to show that $P(k+1)$ is also true. ie): $k\in S \Rightarrow k+1\in S$

\end{enumerate}
By the Principle of Mathematical Induction, $S=\mathbb{N}$

\begin{exmp}
Prove that $\displaystyle\sum_{i=1}^ni=\frac{n(n+1)}{2}$
\end{exmp}
\begin{proof}

Step.1 Let $P(n)$ be the statement that $\displaystyle\sum_{i=1}^ni=\frac{n(n+1)}{2}$

Step.2 Let $n=1$ then $P(1) = 1 = \frac{1(1+1}{2}$. Hence $P(1)$ is true.

Step.3 Assume that $P(k)$ is rue for some $k$
$$P(k) \frac{k(k+1}{2}$$

Step.4 
\begin{align*}
    \sum_{i=1}^{k+1} i & = \sum_{i=1}^k i + (k+1) \\
    & = \frac{k(k+1)}{2}+(k+1)\\
    & = \frac{(k+1)(k+2)}{2}
\end{align*}
Hence $P(k+1)$ is true

Step.5 By Principle of Mathematical Induction, $P(n)$ is true for all $n\in \mathbb{N}$
\end{proof}

\begin{exmp}
Prove that $3^n+4^n$ is divisible by $7$ for every odd $n$
\end{exmp}

\begin{proof}
Let $P(k)$ be the statement that $3^{2k-1}+4^{2k-1}$ is divisible by 7.

Base case: $k=1$, $P(1)$ is true.

Inductive Step: Assume $P(j)$ is true.

\begin{align*}
    &3^{2(j+1)-1}+ 4^{2(j+1)-1}\\
    =& 9(3^{2j-1})+16(4^{2j-1})\\
    =& 9(3^{2j-1}+4^{2j-1}) + 7(4^{2j-1})
\end{align*}
Hence $P(j+1)$ is true.

By Principle of Mathematical Induction, $P(k)$ is true for all $n$

\end{proof}



