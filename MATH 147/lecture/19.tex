\lecture{Oct. 26}

Written Assignment 3 Due Wed, Nov. 9

\begin{thm}[Fundamental Trig Limit]
\[\lim_{x\to 0}\frac{\sin x}{x} = 1\]
\end{thm}
\begin{proof}
Note that $f(x)$ is even. Hence we need only $\lim_{x\to 0^+} \frac{\sin x}{x} = 1$

\begin{center}
\begin{tikzpicture}[scale=0.26]
      \draw[->] (-9,0) -- (9,0) node[right] {$x$};
      \draw[->] (0,-8) -- (0,8) node[above] {$y$};
      \draw[] (0,0) circle (6);
      \draw (0,0) --  (5.196,3) -- (5.196,0);
      \filldraw[black] (2.7,1) circle (0pt) node[anchor=west] {$R_1$};
      \draw[->] (11,0) -- (29,0) node[right] {$x$};
      \draw[->] (20,-8) -- (20,8) node[above] {$y$};
    \draw[] (20,0) circle (6);
    \draw (20,0) -- (25.196,3);
      \filldraw[black] (23,1) circle (0pt) node[anchor=west] {$R_2$};
      \draw[->] (31,0) -- (49,0) node[right] {$x$};
      \draw[->] (40,-8) -- (40,8) node[above] {$y$};
    \draw[] (40,0) circle (6);
    \draw (40,0) -- (46,3.464) -- (46,0);
      \filldraw[black] (43,1) circle (0pt) node[anchor=west] {$R_3$};
\end{tikzpicture}
\end{center}

We have $R_1 = \sin x \cos x / 2$, $R_2 = x/2$ and $R_3 = \sin x / (2 \cos x)$.

Since $R_1\leq R_2 \leq R_3$, we get \[\cos x \leq \frac{x}{\sin x} \leq \frac{1}{\cos x}.\]

Hence \[\cos x \leq \frac{\sin x}{x} \leq \frac{1}{\cos x}.\]

By Squeeze Theorem, $\lim_{x\to 0^+} \frac{\sin x}{x} = 1$.

\end{proof}

\begin{exmp}
Find \[\lim_{x\to 0} \frac{\sin 3x}{\sin 4x}\]
\end{exmp}
\begin{solution}
\begin{align*}
    \lim_{x\to 0} \frac{\sin 3x}{\sin 4x} = & \lim_{x\to 0} \frac{\sin 3x}{3}\cdot \lim_{x\to 0} \frac{4}{\sin 4x} \cdot \frac{3}{4} \\
    = & 1 \cdot 1 \cdot \frac{3}{4} \\
    = & \frac{3}{4}
\end{align*}
\end{solution}

\begin{exmp}
Find \[\lim_{x\to 0} \frac{\tan x}{x}.\]
\end{exmp}

\begin{exmp}
Find \[\lim_{x\to 0} \frac{\tan \pi x}{\sin 2x}.\]
\end{exmp}


\topic{Asymptotes and Limits at $\infty$}

\begin{defn}
We say that $L$ is the limit as x approaches infinity of $f(x)$ if for every $\epsilon > 0$, there exists $M>0$ such that if $x\geq M$, then $\abs{f(x) - L}< \epsilon$. We write \[\lim_{x\to\infty} f(x) = L.\]
\end{defn}

\begin{exmp}
If $f(x) = 1/x$, then $\lim_{x\to\infty} f(x) = 0$.
\end{exmp}

\begin{note}
Arithmetic Rules, Sequential Characterization and Squeeze Theorem carry through.
\end{note}

\begin{thm}[Fundamental Log Limit]
\[\lim_{x\to\infty}\frac{ln(x)}{x} = 0\]
\end{thm}

\begin{proof}
\[\frac{ln(x)}{x} = \frac{2ln(x^{1/2})}{x^{1/2}\cdot x^{1/2}} = \frac{2ln(x^{1/2})}{x^{1/2}} \cdot \frac{1}{x^{1/2}} <\frac{2}{x^{1/2}}\]
By squeeze theorem, $\lim_{x\to\infty}\frac{ln(x)}{x} = 0$
\end{proof}

\begin{exmp}
Find \[\lim_{x\to\infty} \frac{ln(x)}{x^{1/100}}\]
\end{exmp}

\begin{note}
\[\lim_{x\to\infty} \frac{ln(x)}{x^p} = 0 \text{ if } p>0\]
\end{note}

\begin{exmp}
\[\lim_{x\to\infty} \frac{x^p}{e^x} = 0\]
\end{exmp}