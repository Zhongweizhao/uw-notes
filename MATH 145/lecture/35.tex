\lecture{Nov. 14}

\begin{thm}[Linear Congruence Theorem]
Let $n\in\mathbb{Z}^+$, let $a,b\in\mathbb{Z}$, let $d = gcd(a,n)$. Consider the equation \[ax =b \mod n\]
\begin{enumerate}
\item The equation $ax = b\mod n$ has a solution $x\in\mathbb{Z}$ if and only if $d\mid b$
\item If $x = u$ is a solution (so that $au =b \mod n$), then the general solution is \[x = u + k\frac{n}{d} \text{ for } k\in\mathbb{Z}.\]
\end{enumerate}
\end{thm}

\begin{thm}[Chinese Remainder Theorem]
Let $n,m\in\mathbb{Z}^+$ and let $a,b\in\mathbb{Z}$. Then the pair of congruences \[x = a \mod n\] \[x = b\mod m\] has a solution $x\in\mathbb{Z}$ if and only if $d\mid (b-a)$ where $d = gcd(m,n)$, and if $x=u$ is one solution to the pair of congruences then the general solution is $x=u\mod l$ where $l = lcm(n,m)$.
\end{thm}

\begin{proof}
Suppose the pair of congruences has a solution. Choose a solution $x\in\mathbb{Z}$ (so we have $x=a\mod n$ and $x=b\mod m$). Since $x=a\mod n$, we can choose $s$ so that $x = a + ns$, and since $x = b\mod m$, we can choose $t$ so that $x= b + mt$. Then $a+ns = b+mt$, so $ns-mt = b-a$. By the Linear Diophantine Equation Theorem, for $d = gcd(m,n)$, we have $d\mid (b-a)$.

Conversely, suppose that $d\mid (b-a)$. By the Linear Diophantine Equation Theorem we can choose $s,t\in\mathbb{Z}$ so that $ns-mt = b-a$. Then $a+ns = b+mt$. Let $x = a+ns$ (so $x=b+mt$). Then since $x = a+ns$ we have $x=a \mod n$. Since $x = b +mt$ we have $x = b\mod m$.

Suppose that $x=u$ is a solution to the pair of congruences. So we have $u = a \mod n$ and $u = b\mod m$. Let $k\in\mathbb{Z}$. Let $x = u + kl$ where $l = lcm(m,n)$. Since $l = lcm(m,n)$, choose $s,t\in\mathbb{Z}$ so that $l = ns = mt$. Since $x = u + kl = u + kns$, we have $x = u \mod n$ so $x = a\mod n$. Similarly we have $x= b\mod m$. Thus $x = u + kl$ is a solution to the pair of congruences.

Conversely, let $x$ be any solution to the pair of congruences. So we have $x= a \mod n$ and $x = b \mod m$. Since $x = a\mod n$ and $u = a\mod n$, we have $x-u = 0 \mod n$, thus $n\mid x-u$. Since $x = b\mod m$ and $u = b\mod m$, we have $x=u = 0\mod m$, so $m\mid x-u$. Since $n\mid (x-u)$ and $m\mid (x-u)$, it follows from the following lemma that $l \mid (x-u)$ since $l = lcm(m,n)$. Since $l\mid (x-u)$ we have $x = u\mod l$ as required.
\end{proof}

\begin{lem}
Let $n,m\in\mathbb{Z}^+$ and let $l = lcm(m,n)$. For every $k\in\mathbb{Z}$, if $n\mid k$ and $m\mid k$ then $l\mid k$.
\end{lem}

\begin{proof}
Let $k\in\mathbb{Z}^+$ with $n\mid k$ and $m\mid k$. Write $k = \prod_{i = 1}^{q} p_i^{m_i}$ where $q\in\mathbb{Z}^+$, the $p_i$ are distinct primes and each $m_i\in\mathbb{Z}^+$. Since $n\mid k$, every prime factor $p$ of $n$ is also a factor of $k$, so we can write $n = \prod_{i = 1}^{q} p_i^{j_i}$ with each $j_i \in\mathbb{N}$. Similarly, we can write $m = \prod_{i = 1}^{q} p_i^{k_i}$ with each $k_i\in\mathbb{N}$. 

Since $n\mid k$ we have $j_i\leq m_i$ for all indices $i$. Since $m\mid k$, we have $k_i\leq m_i$ for all indices $i$. Since $m_i\geq j_i$ and $m_i\geq k_i$, we have $m_i\geq max(j_i,k_i)$. Thus \[\prod_{i = 1}^{q} p_i^{max(j_i,k_i)} \mid \prod_{i = 1}^{q} p_i^{m_i}\] that is \[lcm(m,n) \mid k\]
\end{proof}

\begin{thm}
For \[n = \prod_{i=1}^q p_i^{k_i}\] where $q\in\mathbb{Z}^+$, the $p_i$ are distinct primes, and each $k_i\in\mathbb{Z}^+$, we have 
\[\varphi (n) = \prod_{i=1}^q \varphi(p_i^{k_i}) = \prod_{i=1}^q p_i^{k_i} -p_i^{k_i-1}\]
\end{thm}

\begin{proof}
By induction, it suffices to show that for all $l,m\in\mathbb{Z}^+$ with $gcd(l,m) = 1$, we have $\varphi (lm) = \varphi(l) \varphi(m)$. We shall prove that $\abs{U_{lm}} = \abs{U_l \cdot U_m}$.

Define $F\colon \mathbb{Z}_{lm}\to \mathbb{Z}_l \times \mathbb{Z}_m$  by $F(x) = (x,x)$ for $x\in\mathbb{Z}$ (that is $F(x\mod lm) = (x\mod l, x\mod m)$).
Note that F is well-defined, which means that for all $x,y \in \mathbb{Z}$ if $x= y \mod lm$ then $x = y \mod k$ and $x=y \mod m$ (if $x= y\mod lm$, say $x=y +tlm$ then $ x = y+(tl)m$ so $ x = y mod m$)\\
Note that F is bijective by the (RT indeed F is surjective (onto) because given $a,b\in\mathbb{Z}$ we can solve $x=a\mod l$ and $x = b\mod m$ and then $F(x) = (x\mod l,x\mod m) = (a,b)$ and $F$ is injective by the Chinese Remainder Theorem.

Finally, it remains to show that $F$ restricts to a bijective map \[F\colon U_{lm} \to U_l \times U_m\] that is for all $x\in \mathbb{Z}$, if $gcd(x,lm) = 1$ then $gcd(x,l) = 1$ and $gcd(x,m) = 1$, and if $gcd(x,l) = 1$ and $gcd(x,m) = 1$, then $gcd(x,lm) = 1$.
\end{proof}