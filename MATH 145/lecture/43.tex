\lecture{Nov. 28}

\begin{thm}[Fundamental Theorem of Algebra]
Every non-constant polynomial with coefficients in $\mathbb{C}$ has a root in $\mathbb{C}$.

Consequently, every non-constant $f(z) \in \mathbb{C}[z]$ can be expressed as \[f(z) = c \prod_{i=1}^n (z - a_i)\] where each $a_i \in \mathbb{C}$ and $0\neq c \in \mathbb{C}$. Alternatively every $f(z)$ of degree $n$ can be expressed as \[f(z) = c\prod_{i=1}^l (z-a_i)^{k_i}\] where $l\in\mathbb{Z}^+$, the $a_i$ are distinct complex numbers, $k_i \in\mathbb{Z}^+$ with $\sum_{i=1}^l k_i =n$.
\end{thm}

\begin{note}
Let $f(x) \in \mathbb{R}[x]$ say \[f(x) = c_0 + c_1 x + \cdots + c_nx^n, \,\ c_n \neq 0\]
If $\alpha \in \mathbb{C}$ then \[f(\overline{\alpha}) =\overline{f(\alpha)}\] It follows that if $f(\alpha) = 0$ then $f(\overline{\alpha}) = 0$. Consequently, every non-constant polynomial $f(x) \in \mathbb{R}[x]$ factors in $R[x]$ into a product of linear and quadratic terms.
\end{note}