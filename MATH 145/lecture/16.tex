\lecture{Oct. 5}



\begin{thm}
Let $F(n)$ be a statement about an integer n. Let $m\in \mathbb{Z}$

Suppose $F(m)$ is true 

Suppose that for all $k\geq m$, if $F(k)$ is true then $F(k+1)$ is true 

Then $F(n)$ is true for all $n\geq m$

\end{thm}

\topic{Proof Method}
Let $F(n)$ be a statement about an integer and let $m\in \mathbb{Z}$

To prove $F(n)$ is true for all $n\geq m$, we can do the following.

\begin{enumerate}
\item Prove that $F(n)$ is true
\item Let $k\geq m$ be arbitrary and suppose, inductively, that $F(k)$ is true
\item Prove $F(k+1)$ is true
\end{enumerate}

Alternatively, suppose $F(k-1)$ prove $F(k)$

\topic{A slightly different proof method}
To prove that $F(n)$ is true for all $n\geq m$ we can do the following:
\begin{enumerate}
\item Prove that $F(m)$ is true and that $F(m+1)$ is true
\item Let $k\geq m+2$ be arbitrary and suppose that $F(k-1)$ and $F(k-2)$ are true
\item Prove that $F(k)$ is true
\end{enumerate}

\topic{Another Proof Method}
we can prove that $F(n)$ is true for all $n\geq m$ as follows.

\begin{enumerate}
\item Let $n\geq m$ be arbitrary and suppose that $F(k)$ is true for all $k$ with $m\leq k < n$
\item prove that $F(n)$ is true.
\end{enumerate}


\begin{thm}

\textbf{Strong Mathematical Induction}
Let $F(n)$ be a statement about an integer n and let $m\in \mathbb{Z}$

Suppose that for all $n\geq m$, if $F(k)$ for all $k\in \mathbb{Z}$ with $m\leq k < n$, then $F(n)$ is true.

Then $F(n)$ is true for all $n\geq m$.

\end{thm}

\begin{proof}
Let $G(n)$ be a statement ``F(n) is true for all $k\in \mathbb{Z}$ with $m\leq k < n$"

Note that $G(m)$ is true \underline{vacuously}. (since there is no value of $k\in \mathbb{Z}$ with $m\leq k < n$)

Let $n\geq m$ be arbitrary.

Suppose $G(n)$ is true, that is ``$F(n)$ is true for all $k\in\mathbb{Z}$ with $m\leq k < n$"

Since $F(n)$ is true for all $k\in\mathbb{Z}$ with $m\leq k < n$, then $F(n)$ is true for all $k\in\mathbb{Z}$ with $m\leq k < n+1$. In other words, $G(n+1)$ is true.


Now let $n\geq m$ be arbitrary. Since $G(k)$ is true for all $k\geq m$, in particular $G(n+1)$. In other words, $F(k)$ is true for all k with $m\leq k < n+1$. In particular $F(n)$ is true

Since $n\geq m$ was arbitraty, $F(n)$ is true for all $n\geq m$.
\end{proof}


\begin{exmp}
Let $(x_n)_{n\geq 0}$ be the sequence which is defined recursively by $x_0 = 2$, $x_1 = 2$ and $x_n = 2 x_{n-1} + 3 x_{n-2}$ for all $x\geq 2$

Find a closed formula for $x_n$
\end{exmp}

\begin{solution}
Observe that $x_n = 3^n + (-1)^n$

When $n = 0$, $x_0 = 2 $ and $3^0 + (-1)^0 = 2$, so $x_n = 3^n + (-1)^n$ is true when $n = 0$

When $n = 1$, $x_1 = 2 $ and $3^1 + (-1)^1 = 2$, so $x_n = 3^n + (-1)^n$ is true when $n = 1$

Let $n \geq 2$ be arbitrary.

Suppose that $x_{n-1} = 3^{n-1}+(-1)^{n-1}$ and $x_{n-2} = 3^{n-2}+(-1)^{n-2}$

\begin{align*}
    x_n & = 2x_{n-1}+3x_{n-2} \\
    & = 2(3^{n-1}+(-1)^{n-1}) + 3(3^{n-2}+(-1)^{n-2})\\
    & = 9^{n-2}+(3-2)(-1)^{n-2} \\
    & = 3^{n}+(-1)^{n}
\end{align*}

By induction, $x_n = 3^n + (-1)^n$ for all $n\geq 0$


\end{solution}

\topic{Binomial Theorem}

\begin{defn}
For $n,k\in\mathbb{N}$ with $0\leq k\leq n$
\[
\binom{n}{k} = \frac{n!}{k!(n-k)!} = \frac{n(n-1)(n-2)\dots (n-k+1)}{k!}
\]

\end{defn}




