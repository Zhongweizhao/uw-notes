\lecture{Sept. 20}

\begin{exmp}

Let $F$ and $G$ be formulas 

Determine whether $$\{F\to G, F\vee G\}\vDash F\wedge G$$

\end{exmp}

\begin{solution}
We make a truth table
\begin{center}
\begin{tabular}{c|c|c|c|c}
    $F$ & $G$ & $F\to G$ & $F\vee G$ & $F\wedge G$ \\ \hline
    1 & 1 & 1 & 1 & 1 \\\hline
    1 & 0 & 0 & 1 & 0 \\\hline
    0 & 1 & 1 & 1 & 0 \\\hline
    0 & 0 & 1 & 0 & 0 \\
\end{tabular}
\end{center}

\end{solution}

\begin{rem}

It appears from row 3 that the argument is not valid.

But in fact, the argument may or may not be valid, depending on the formulas $F$ and $G$.

For example, if $F$ is a tautology and $G$ is any formula, the argument is valid.

Or if $F=G$ then the argument is valid.

\end{rem}

\topic{First-Order Language}

\topic{Symbol Set}
\begin{defn}
A First-Order Language is determined by its symbol set. The symbol set includes symbols from the common symbol set $$\{\neg ,\wedge ,\vee ,\to,\leftrightarrow,=,\forall,\exists,(,),,\}$$
along with some variable symbols such as $$x,y,z,u,v,w,\dots$$

The symbol $=$ is read as "equals". The symbols $\forall$, $\exists$ are called quantifier symbols. The symbol $\forall$ is read as "for all" or "for every", and the symbol $\exists$ is read in "for some" or "there exists".

The symbol set can also include some additional symbols which can include 
\begin{enumerate}
\item constant symbols $$a,b,c,\emptyset,0,i,e,\pi,\dots$$
\item function symbols $$f,g,h,\cup,\cap,+,\times,\dots$$
\item relation symbols $$P,Q,R,\in,\subset,\subseteq,<,>,=,\dots$$
\end{enumerate}
\end{defn}
The variable and constant symbols are intended to represent elements in a certain set or class $u$ called the universal set or the universal class. The universal set or class is often understood from the context.


\topic{Function}
\begin{defn}
A unary function $f$ from a set $u$ is a function $f\colon u\to U$ (for every $x\in u$ there is a unique element $y=f(x)\in U$)

A binary function $g$ on $u$ is a function $g\colon u^2\to U$ where $u^2 = u\times u$ (for every $x,y\in u$ there is a unique element $z=g(x,y)\in U$)
\end{defn}

Some binary function symbols are used with infix notation, which means that we write $g(x,y)$ as $xgy$ or as $(xgy)$ 

\begin{exmp}

$+$ is a binary function on $\mathbb{N}$ written with infix notation. So we write $+(x,y)$ as $x+y$ or as $(x+y)$.
\end{exmp}


\topic{Relation}
\begin{defn}
A unary relation $P$ on a set $u$ is a subset $P\subseteq U$. For $x\in u$, we write $P(x)$ to indicate that $x\in P$

A binary relation $R$ on $u$ is a subset of $U^2$. We write $R(x,y)$ to indicate that $(x,y)\in R$
\end{defn}
Sometimes a binary relation symbol $R$ is used with indix notation which means that we write $R(x,y)$ as $xRy$.


\begin{exmp}

$<$ is a binary relation on $\mathbb{N}$, which means that $<\subseteq \mathbb{N}^2$ and it is used with infix notation, So we write $<(x,y)$ as $x<y$

Also, the symbol = is a binary relation symbol written with infix notation.

\end{exmp}

\begin{rem}
$(P\wedge Q)$ can be written with infix notation as $\wedge PQ$, which is also called polish notation.
\end{rem}

\topic{Term}
\begin{defn}
In a first-order language, a term is a non-empty finite string of symbols from the symbol set which can be obtained by applying the following rules.
\begin{enumerate}
\item Every variable symbol is a term and every constant symbol is a term.
\item if $f$ is a unary function symbol and $t$ is a term, then the string $f(t)$ is a term
\item if $g$ is a binary function symbol and $s$ and $t$ are terms, the the string $g(s,t)$ (or the string $sgt$) is a term.
\end{enumerate}
\end{defn}

\begin{exmp}

The following strings are terms.
\begin{itemize}
\item $u$
\item $u\cap v$
\item $u\cap (v \cap \emptyset)$
\item $x$
\item $x+1$
\item $g(x,f(y+1))$
\end{itemize}

Each term represents an element in the universal set or class $u$

\end{exmp}

\topic{Formula}
\begin{defn}
A formula is a non-empty finite string of symbols which can be obtained using the following rules.
\end{defn}
\begin{enumerate}
\item if $P$ is a unary relation symbol and $t$ is a term then the string $P(t)$ is a formula. (in standard mathematical language we would write $P(t)$ as $t\in P$)
\item if $R$ is a binary relation symbol and $s$ and $t$ are terms then the string $R(s,t)$ is a formula (or $sRt$)
\item if $F$ is a formula, then so is the string $\neg F$
\item if $F$ and $G$ are formulas then so is each of the strings $F\wedge G$, $F\vee G$, $F\to G$, $F\leftrightarrow G$
\item if $F$ is a formula and $x$ is a variable symbol, then the string $\forall x F$ and $\exists x F$ are both formulas
\end{enumerate}

\textbf{Examples: }
Each of the following strings is formula
\begin{itemize}
\item $u\subseteq R$
\item $\forall u \enspace \emptyset \in u$
\item $f(x)<x+1$
\item $x=g(y,z+1)$
\end{itemize}

A formula is a formal way of expressing a mathematical statement about element in $u$, and about functions and relations on $u$.


\begin{rem}
In standard mathematical language, we continually to add new notations which we allow ourselves to use.
\end{rem}

\begin{exmp}

$\displaystyle\frac{x+1}{y}$ could be written as $(x+1)/y$

$\displaystyle\sum_{k=1}^n \frac{1}{k}$ could be written as .... (a very long formula)
\end{exmp}

