\lecture{Oct. 2}

\begin{thm}[Linear Diophantine Equation Theorem]
Let $a,b,c\in\mathbb{Z}$ with $(a,b)\neq (0,0)$. Let $d = gcd(a,b)$. Consider the equation \[ax+by =c.\] The equation has a solution $(x,y)$ with $x,y\in\mathbb{Z}$ if and only if $d\mid c$. In this case if $(u,v)$ is a solution with $u,v\in\mathbb{Z}$, then the general solution is \[(x,y) = (u,v) + k (-\frac{b}{d},\frac{a}{d})\] 
\end{thm}

\begin{proof}
Suppose that the equation has a solution. Choose $x,y\in\mathbb{Z}$ so that $ax+by = c$. Since $d\mid a$ and $d\mid b$, $d\mid ax+by$, so $d\mid c$.

Conversely, Suppose that $d\mid c$, say $c = dl$. Use EA with BS to obtain $s,t\in\mathbb{Z}$ such that \[as+bt = d\]. Then \[asl+btl=dl=c.\] So we have \[ax+by = c\] with $x=sl$ and $y=tl$

Suppose that $d\mid c $ and suppose that $u,v\in\mathbb{Z}$ with $au+bv = c$. We need to show that 
\begin{enumerate}
\item for all $k\in\mathbb{Z}$, if we let $(x,y) = (u,v) + k (-\frac{b}{d},\frac{a}{d})$, then $ax+by = c$
\item for all $x,y\in\mathbb{Z}$, if $ax+by = c$, then there exists $k\in\mathbb{Z}$ such that $(x,y) = (u,v) + k (-\frac{b}{d},\frac{a}{d})$
\end{enumerate}

To prove 1, let $k\in\mathbb{Z}$ and let $(x,y) = (u,v) + k (-\frac{b}{d},\frac{a}{d})$, that is $x = u-k\frac{b}{d}$ and $y = v+k\frac{a}{d}$. Then
\begin{align*}
    ax+by = & a (u-k\frac{b}{d}) + b(v+k\frac{a}{d}) \\
    =& au + bv - k \frac{ab}{d}+ k \frac{ab}{d} \\
    = & au+bv \\
    = & c
\end{align*}

To prove 2, let $x,y\in\mathbb{Z}$. Suppose $ax+by = c$. Since $ax+by = c$ and $au+bv = c$, \[a(x-u) + b(y-v) = 0\] so \[\frac{a}{d}(x-u) = -\frac{b}{d}(y-v)\] and note that $\frac{a}{d} \in\mathbb{Z}$ and $\frac{b}{d}\in\mathbb{Z}$. It follows that \[\frac{a}{d}\mid (y-v).\] Choose $k\in\mathbb{Z}$ so that \[y-v = k\frac{a}{d}.\] Since $y-v = k\frac{a}{d}$ and \[\frac{a}{d}(x-u) = -\frac{b}{d} (y-v)\] we have \[\frac{a}{d}(x-u) = -\frac{b}{d}k\frac{a}{d}\] so \[x-u = -k\frac{b}{d}.\] So we have \[x=u-k\frac{b}{d} \text{ and } y = v + k \frac{a}{d} \qedhere\]

\end{proof}

\begin{thm}[Unique Prime Factorization]
    Every integer $n\geq 2$ can be expressed uniquely in the form \[n=\prod_{i=1}^l p_i = p_1p_2\cdots p_l\] for some $l\in\mathbb{Z}^+$ and some primes $p_1,p_2,\cdots,p_l$ with $p_1\leq p_2\leq \cdots \leq p_l$.
    
    Alternatively, every integer $n\geq 2$ can be written uniquely in the form \[n = \prod_{i=1}^l p_i^{k_i}\] with $l\in\mathbb{Z}^+$ and $p_i$ are distinct primes with $p_1< p_2< \cdots < p_l$ and each $k_i\in\mathbb{Z}^+$.
    
    Alternatively, given an integer $n\geq 1$ if every prime factor of n is included in the set $\{p_1,p_2,\cdots , p_l\}$ where the $p_i$ are distinct primes, then $n$ can be written uniquely in the form \[n = \prod_{i=1}^l p_i^{k_i}\] with each $k_i \in\mathbb{N}$ 
    
    When \[n = \prod_{i=1}^l p_i^{k_i}\] where the $p_i$ are distinct primes and each $k_i \in\mathbb{N}$, the positive divisor of $n$ are the integers $a$ of the form \[a = \prod_{i=1}^l p_i^{d_i}\] such that $0\leq d \leq k_i$ for all indices $i$. 
    
\end{thm}

\begin{thm}
    The number of positive divisors of $n$ is \[\tau (n) = \prod_{i=1}^l (k_i + 1)\]
    
    The sum of the positive divisors of $n$ is \[\sigma (n) = \prod_{i=1}^l 
    \frac{p_i^{n+1}-1}{p-1}.\]
\end{thm} 

\begin{thm}
    The product of all the positive divisors of $n$ is \[p(n) = n^{\tau (n)/2}\]
\end{thm}

\begin{proof} 
    exercise
\end{proof}

\begin{defn}
    For $n = \prod_{i=1}^l p_i^{k_i}$ the exponent of $p$ in $n$
    \begin{equation*}
        \begin{cases}
            k_i & \text{ if } p = p_i\\
            0 & \text{ if } p\not\in\{p_1,p_2,\cdots ,p_l\}
        \end{cases}
    \end{equation*}
\end{defn}
