\lecture{Sept. 21}

\topic{Recap}

Symbol Set

Term

Formula

\topic{First-Order Language}


\begin{defn}
In the language of first-order number theory, we allow us to use the following additional symbols:

\[\{0,1,+,\times,<\}\]

Unless otherwise stated, we do not allow ourselves to use any other additional symbols.
\end{defn}

\begin{exmp}
Express each of the following statement as formulas in the language of first-order number theory.
\begin{enumerate}
\item[a)] x is a factor of y
\item[b)] x is a prime number
\item[c)] x is a power of 3
\end{enumerate}

\end{exmp}

\begin{solution}

We take he universal set to be \(\mathbb{Z}\).

\begin{enumerate}
\item[a)] \(\exists z\in \mathbb{Z}\enspace y=x\times z\)
\item[b)] \(1<x\wedge \forall y (\exists z \enspace x=y\times z \to ((y=1\vee y=x)\vee(y+1=0\vee y+x=0) ))\) \\
	\(1<x\wedge \forall y ((1<y\wedge \exists z \enspace x=z\times y)\to y=x)\)
    
\item[c)] \((0<x )\wedge\) the only prime factor of x is 3\\
\(\iff (0<x) \wedge \forall y\in \mathbb{Z} ((\text{y is prime}\wedge \text{y is a factor of x})\to y=3)\)\\
\(\iff (0<x) \wedge \forall y\in \mathbb{Z} ((1<y\wedge\exists z \enspace x=y\times z) \to \exists z\enspace y=((z+z)+z))\)

\end{enumerate}
\begin{align*}
x=-y &\iff x+y=0\\
x=y-z &\iff x+z=y
\end{align*}


\end{solution}

\begin{rem}
The two minus signs in the two equations above are different.
\end{rem}



\begin{exmp}
Express the following statements about a function \(f\colon \mathbb{R}\to\mathbb{R}\) as formulas in first-order number theory after adding the function symbol \(f\) to the symbol set.
\end{exmp}


\begin{enumerate}
\item[a)] \(f\) is surjective (or onto)
\item[b)] \(f\) is bijective (or invertible)
\item[c)] \(\lim_{x\to u} f(x) = v\)
\end{enumerate}

\begin{solution}

\begin{enumerate}
\item[a)] \(\forall y\in\mathbb{R}\enspace \exists x\in \mathbb{R}\enspace y=f(x)\)
\item[b)] \(\forall y\in\mathbb{R}\enspace \exists! x\in \mathbb{R}\enspace y=f(x)\)\\
\(\iff \forall y\in\mathbb{R}\enspace (\exists x\in \mathbb{R}(y=f(x)\wedge \forall z (y=f(z)\to z=x)))\)
\item[c)] \(
\forall \epsilon > 0 \enspace \exists  \delta > 0 \enspace \forall x\in \mathbb{R} (0<|x-v|<\delta \to |f(x)-v|<\epsilon)
\)\\
\(
\iff 
\forall \epsilon (0<\epsilon \to \exists \delta (0<\delta \wedge \forall x ((\neg x = u \wedge (u < x+\delta \wedge x<u+\delta)) \to (v<f(x) + \epsilon \wedge f(x)<v+\epsilon))))
\)
\end{enumerate}
\end{solution}











