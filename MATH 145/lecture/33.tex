\lecture{Nov. 9}

\topic{Powers Modulo $n$}

$\mod 5$ in $\mathbb{Z}_5$
\[
\begin{array}{cccccc}
    x & 0 & 1 & 2 & 3 & 4 \\
    x^2 & 0 & 1 & 4 & 4 & 1 \\
    x^3 & 0 & 1 & 3 & 2 & 4 \\
    x^4 & 0 & 1 & 1 & 1 & 1 \\
    x^5 & 0 & 1 & 2 & 3 & 4 \\
\end{array}
\]

$\mod 7$ in $\mathbb{Z}_7$
\[
\begin{array}{cccccccc}
    x & 0 & 1 & 2 & 3 & 4 & 5 & 6 \\
    x^2 & 0 & 1 & 4 & 2 & 2 & 4 & 1 \\
    x^3 & 0 & 1 & 1 & 6 & 1 & 6 & 6 \\
    x^4 & 0 & 1 & 2 & 4 & 4 & 2 & 1 \\
    x^5 & 0 & 1 & 4 & 5 & 2 & 3 & 6 \\
    x^6 & 0 & 1 & 1 & 1 & 1 & 1 & 1 \\
    x^7 & 0 & 1 & 2 & 3 & 4 & 5 & 6 \\
\end{array}
\]

$\mod 20$ in $\mathbb{Z}_{20}$
\[
\begin{array}{ccccccc}
    x & 0 & 1 & 2 & 3 & 4 & 5\\
    x^2 & 0 & 1 & 4 & 9 & 16 & 5\\
    x^3 & 0 & 1 & 8 & 7 & 4 & .\\
    x^4 & 0 & 1 & 16 & 1 & . & .\\
    x^5 & 0 & 1 & 12 & . & . & .\\
    x^6 & 0 & 1 & 4 & 9 & 16 & 5\\
\end{array}
\]


\begin{conj}
    for $n \in \mathbb{Z}^+$
    $2^{n-1}$ mod n $\iff$ n is prime. This is false
\end{conj}

\begin{thm}[Fermat's Little Theorem]
let $p$ be a prime then
    \begin{enumerate}
        \item for all $a\in \mathbb{Z}$ such that $gcd(a,p) = 1$, \[a^{p-1} = 1 \mod p\]
        \item for all $a\in \mathbb{Z}$, \[a^p  = a \mod p\]
    \end{enumerate}
\end{thm}

\begin{proof}
\begin{enumerate}
    \item Let $p$ be prime. Let $a\in\mathbb{Z}$ with $gcd(a,p) = 1$. Then $a$ i invertible in $\mathbb{Z}_p$. Define $F\colon \mathbb{Z}_p\to \mathbb{Z}_p$, by $F(x) = ax$. Note that $F$ is bijective with inverse function $G\colon\mathbb{Z}_p\to \mathbb{Z}_p$, given by $G(x) = a^{-1}x$. Aso note that $F(0) = 0$. So $F$ gives a bijection $F\colon U_p\to U_p$. That is $F\colon \{1,2,3\cdots, p-1\}\to \{1,2,3\cdots, p-1\}$. In other words, \[\{1,2,3\cdots, p-1\} = \{1\cdot a, 2\cdot a,\cdots ,(p-1)\cdot a\}\] Thus \[(1\cdot a)(2\cdot a)\cdots ((p-1)\cdot a) = 1\cdot 2\cdot 3\cdots (p-1)\] therefore \[a^{p-1} = 1\] in $\mathbb{Z}_p$.

    \item Let p be prime. Let $a\in\mathbb{Z}$. If $gcd(a, p) = 1$ in so $p\not\mid a$ then by 1, we have $a^{p-1} = 1$ in $\mathbb{Z}_p$. So we can multiply both sides by $a$ to get \[a^p = a\] in $\mathbb{Z}_p$ If $gcd(a,p)\neq 1$ so $gcd(a,p)=p$ so $p\in a$, then $a = 0 \in \mathbb{Z}$ so $a^p = 0^p = 0 = a \in \mathbb{Z}_p$
\end{enumerate}
\end{proof}

\begin{thm}[Euler-Fermat Theorem]
Let $n \in \mathbb{Z}^+$. For all $a\in\mathbb{Z}$ with $gcd(a,n) = 1$, \[a^{\varphi (n)} = 1 \mod n\]
\end{thm}
\begin{proof}
Let $n\in\mathbb{Z}^+$. When $n=1$ we have $\varphi (n) = 1$. So for $a\in \mathbb{Z}$, $a^{\varphi{n}} = a^{1} = a$.

Suppose $n\geq 2$. Let $a\in\mathbb{Z}$ with $gcd(a,n) = 1$. Since $gcd(a,n) = 1$, we have $a\in U_n$. The function $F\colon U_n\to U_n$ given by $F(x) = ax$ is bijective with inverse $G\colon U_n\to U_n$, given by $G(x) = a^{-1}x$. So the set $U_n$ is equal to the set $\{ax \mid x\in U_n\}$. It follows that \[\prod_{x\in U_n} (ax) = \prod_{x\in U_n} x\] then \[a^{\varphi(n)}=1\] in $U_n$.
\end{proof}

\begin{thm}
Let $G$ be a finite commutative group. Then for all $a\in G$, \[a^{\abs{G}} = 1\] (where for a finite set S, $\abs{S}$ denotes the number of elements in S)
\end{thm}


\topic{Divisibility Test in Base 10}

Let $n = \sum_{i=0}^m d_i 10^i$ where each $d_i\in\{1,2,\cdots , 9\}$.

Noto that $2 \mid 10$, so $2^k\mid 10^k$ and $2^k \mid 10^l$ for all $l\geq k$.So \[10^l = 0 \in \mathbb{Z}_{2^k} \text{ when } l\geq k\]

So in $\mathbb{Z}_{2^k}$,
\[n = \sum_{i=0}^m d_i 10^i = \sum_{i = 0}^{k-1} d_i 10^i\]

So $2^k \mid n \Longleftrightarrow 2^k$ divides the tailing $k$-digit number of $n$.


Similarly we have Divisibility Test for $3,9,11$.
