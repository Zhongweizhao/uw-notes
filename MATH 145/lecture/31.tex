\lecture{Nov. 7}

\begin{defn}[Representative]
For $x,a\in S$ with $\sim$m when $x\in [a]$, that is when $[x] = [a]$, we say that $x$ is a representative of the equivalence class $[a]$.
\end{defn}

\begin{defn}
Let $n\in\mathbb{Z}^+$. Define a relation on $\mathbb{Z}$ as follows. For $a,b\in\mathbb{Z}$, we define 
\begin{align*}
    a\sim b \Longleftrightarrow & n\mid (a-b) \\
    \Longleftrightarrow & a-b = kn \text{ for some } k\in\mathbb{Z} \\
    \Longleftrightarrow & a= b + kn \text{ for some } k\in\mathbb{Z}
\end{align*}

More commonly, we write \[a = b \mod  n\] when $a\sim b$, and we say that $a$ is equal (or equivalent or congruent) to b modulo n.
\end{defn}

Note that this relation is an equivalence class because for $a,b,c\in\mathbb{Z}$,
\begin{enumerate}
    \item $a\sim a$ since $a = a+0\cdot n$
    \item if $a\sim b$, say $a= b + k\cdot n$ with $k\in\mathbb{Z}$, then $b = a + (-k)\cdot n$, so $b\sim a$
    \item if $a\sim b$, and $b\sim c$, say $a=b+kn$ and $b = c+ln$ with $k,l\in\mathbb{Z}$, then $a= c + (l+k)n$, so $a\sim c$
\end{enumerate}

\begin{defn}
We define the set of integers modulo n to be the quotient set \[\mathbb{Z}_n = \mathbb{Z}/\sim = \{[a]\mid a\in\mathbb{Z}\}\] where 
\begin{align*}
    [a] = & \{x\in\mathbb{Z}\mid x\sim a\} \\
    = & \{x\in \mathbb{Z} \mid x= a \mod  n\} \\
    = & \{x\in\mathbb{Z} x = a+kn \text{ for some } k\in\mathbb{Z}\}\\
    =& \{\cdots,a-2n,a-n,a,a+n,a+2n,\cdots\}
\end{align*}
\end{defn}

\begin{rem}
Note that for $n\in\mathbb{Z}^+$ and for $a,b\in\mathbb{Z}$, we have $a=b \mod  n$ if and only if $a$ and $b$ have the same remainder when divided by $n$. That is if $a = qn+r$ with $0\leq r < n$ and $b = pn+s$ with $0\leq s < n$, then $a= b \mod  n \Longleftrightarrow r=s$
\end{rem}
\begin{proof}
Suppose $a = qn+r$ with $0\leq r < n$ and $b = pn+s$ with $0\leq s < n$. Suppose that $a=b \mod  n$, so that $n\mid (a-b)$. We have $a-b = (q-p)n + (r-s)$. Since $n\mid (a-b)$, we have $n\mid (r-s)$. If $r\neq s$ so $r-s \neq 0$ then since $n\mid (r-s)$ we have $n\leq \abs{r-s}$. But since $0\leq r < n$ and $0\leq s < n$, we have $r-s < n-s \leq n-0 = n$, and $s-r <n-r \leq n-0 = n$, so $\abs{r-s} < n$, giving a contradiction. Thus $r=s$.

Conversely Suppose that $r=s$, then $a-b = (q-p)n + (r-s) = (q-p)n$, so $n\mid (a-b)$, hence $a=b \mod  n$.
\end{proof}

Since the possible remainders $r$ with $0\leq r < n$ are $0,1,2,\cdots , n-1$, it follows that \[\mathbb{Z}_n = \{[0],[1],\cdots ,[n-1]\}\] and the elements listed in the set are distinct (so that $\mathbb{Z}_n$ has exactly $n$ elements).

Often, for $n\in\mathbb{Z}^+$ and $a\in\mathbb{Z}$ we shall write the element $[a]$ in $\mathbb{Z}_n$ simply as $a\in \mathbb{Z}_n$. So for $a,b\in\mathbb{Z}$ we have 
\begin{align*}
    & a = b \mod n \text{ in }\mathbb{Z} \\
    \Longleftrightarrow & a = b \text{ in }\mathbb{Z}_n
\end{align*}

\begin{thm}
For $n\in\mathbb{Z}$ with $n\geq 2$, $\mathbb{Z}_n$ is a ring using the following operations: for $a,b\in\mathbb{Z}$ we define \[[a] + [b] = [a+b]\] and \[[a]\cdot [b] = [ab].\] 

The zero and identity elements in $\mathbb{Z}_n$ are $[0]$ and $[1]$.
\end{thm}

Let us verify that the operations are well-defined.
\begin{proof}
We need to show that for $a,b,c,d\in\mathbb{Z}$, if $a=c \mod n$ and $b = d \mod n$, then $a+b = c + d \mod n$ and $ab = cd \mod n$.

Let $a,b,c,d \in\mathbb{Z}$, Suppose $a=c \mod n$ and $b = d \mod n$, say $a=c+kn$ and $b = d+ln$, then $a+b = (c+d) + (l+k)n$, so $a+b = c + d \mod n$, and $ab = cd + (cl+kd+kln)n$, so $ab = cd \mod n$.
\end{proof}

It is easy to check that the axioms are satisfied.

For example, for $a,b,c\in\mathbb{Z}$, 
\begin{align*}
    [a] + [0] =& [a + 0]\\
    =& [a]
\end{align*}

\begin{align*}
    [a][1] = & [a\cdot 1]\\
    = & [a]
\end{align*}

\begin{align*}
    [a]([b] + [c]) = & [a][(b + c)]\\
    = & [a(b + c)] \\
    = & [ab + ac] \\
    = & [ab] + [ac]\\
    = & [a][b] + [a][c]
\end{align*}

\begin{thm}[Units Modulo n]
For $a,n\in\mathbb{Z}$ with $n\geq 2$. \[[a] \text{ is invertible in } \mathbb{Z}_n \Longleftrightarrow gcd(a,n) = 1 \text{ in }\mathbb{Z}\]
\end{thm}

\begin{proof}
Suppose $[a]$ is a unit in $\mathbb{Z}_n$. Choose $s\in\mathbb{Z}$ so that $[a][s] = 1$. Then $[as] = 1$ and $as = 1 \mod n$. Say $as = 1 + kn$ with $k\in\mathbb{Z}$, then $as + nt = 1$ with $t = -k$. Thus $gcd(a,n) = 1$.

Conversely, suppose $gcd(a,b) = 1$. Use the Euclidean Algorithm with Back Substitution to find $s,t\in \mathbb{Z}$ such that $as+nt = 1$.Then $as = 1 -nt$. Thus $[as] = [1]$ in $\mathbb{Z}_n$, so $[a][s] = 1$. So $[a]$ is invertible with $[a]^{-1} = [s]$ in $\mathbb{Z}_n$. 
\end{proof}

\begin{exmp}
Determine whether 125 is a unit in $\mathbb{Z}_{471}$ and, if so, find $125^{-1}$.
\end{exmp}