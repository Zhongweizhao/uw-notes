\lecture{Oct. 4}

\begin{defn}
For $a,b \in\mathbb{Z}$, if $a\neq 0$ and $b\neq 0$ then $lcm(a,b)$ is the smallest $m\in\mathbb{Z}^+$  such that $a\mid m$ and $b\mid m$, and $lcm(a,0) =lcm(0,a) = 0$.
\end{defn}


\begin{thm}
Let $a,b\in\mathbb{Z}$. Write \[a = \prod_{i=1}^m p_i^{k_i}\] and \[b = \prod_{i=1}^m p_i^{l_i}\] then \[gcd(a,b) = \prod_{i=1}^m p_o^{min(k_i,l_i)}\] and \[lcm(a,b) = \prod_{i=1}^m p_o^{max(k_i,l_i)}\] and \[gcd(a,b)lcm(a,b) = ab\]
\end{thm}

\topic{Here ends Chapter 3: Factorization in $\mathbb{Z}$}

\topic{Chapter 4: Integers Modulo $n$}

Recall that, informally, $\mathbb{Z}_n = \{0,1,2,\cdots , n-1\}$ and we add and multiply by adding and multiplying in $\mathbb{Z}$ then finding the remainder after dividing by $n$. 

\begin{defn}[Partition]
A partition of a set $S$ is a set $P$ of non-empty disjoint sets whose union is $S$, that is for all $A\in P$, $A\neq \emptyset$, for all $A,B \in P$, if $A\neq B$ then $A\cap B = \emptyset$ and \[\bigcup P = \bigcup_{A\in P} A = S\] or equivalently for all $A\in P$ we have $A \subseteq S$ and for all $a\in S$ we have $a\in A$ for some $A\in P$
\end{defn}

\begin{defn}[Equivalence Relation]
A equivalence relation on a set $S$ is a binary relation $\sim $ on s such that
\begin{enumerate}
\item Reflexivity: for all $a\in S$, $a\sim a$
\item Symmetry: for all $a,b\in S$, if $a\sim b$ then $b\sim a$
\item Transitivity: for all $a,b\in S$, if $a\sim b$ and $b\sim c$ then $a\sim c$
\end{enumerate}
\end{defn}

\begin{defn}[Equivalence Class]
Given an equivalence relation $\sim $ on a set $S$, for $a\in S$, the equivalence class of a (in $S$ under $\sim $) is the set \[a] = \{x\in S \mid x\sim a\}\]
\end{defn}

\begin{thm}[Equivalence Classes Form a Partition]
Let $S$ be a set. Let $\sim $ be an equivalence relation on $S$. Then
\begin{enumerate}
\item for all $a\in S$ we have $a\in [a]$
\item for all $a,b \in S$ we have $[a]=[b] \Leftrightarrow a \sim b \Leftrightarrow a\in [b] \Leftrightarrow b\in [a]$
\item for all $a,b\in S$, if $[a] \neq [b]$ then $[a] \cap [b] = \emptyset$
\end{enumerate}

It follows that \[P = \{[a]\mid a\in S\}\] is a partition of S
\end{thm}

\begin{proof}
\begin{enumerate}
\item for $a\in S$ we have $a\in [a]$ since $a\sim a$
\item Let $a,b \in S$. Suppose $[a]=[b]$. Then $a\in [a]$ and $[a]=[b]$ so $a\in [b]$ so $a\sim b$. Note that $a\in [b] \Leftrightarrow a\sim b$. Suppose that $a\in [b]$ then $a\sim b$. If $x\in [a]$ then $x\sim a$, so we have $x\sim b$ and so $x\in [b]$. Conversely, if $x\in [b]$ so $x\sim b$ then we have $x\sim a$ and so $x\in [a]$. This shows that $[a] = [b]$
\item Let $a,b\in S$. Suppose $[a] \cap [b] \neq \emptyset$. Choose $c\in [a] \cap [b]$. Since $c\in [a]$ we have $[c] = [a]$. Since $c\in [b]$ we have $[c]=[b]$. Thus $[a]=[c]=[b]$.\qedhere
\end{enumerate}
\end{proof}

\begin{defn}[Quotient]
When $\sim$ is an equivalence relation on $S$, the partition $p = \{[a] \mid a\in S\}$ is called the quotient of $S$ by $\sim$ and is denoted by $S/\sim$. So we have \[S/\sim = \{[a] \mid a\in S\}\]
\end{defn}

\begin{exmp}
We construct $\mathbb{Z}$ from $\mathbb{N}$ using a quotient construction.

We define a relation $\sim$ on $\mathbb{N}^2 = \mathbb{N}\times\mathbb{N}$ by defining $(a,b)\sim (c,d) \Leftrightarrow a+d = b+c$. We check that $\sim$ is an equivalence relation. We define \[\mathbb{Z} = \mathbb{N}^2/\sim = \{[(a,b)] \mid a \in \mathbb{N} , \, b \in \mathbb{N}\}\]

For $n\in\mathbb{N}$ we write \[n=[(n,0)] = \{(n,0),(n+1,1),\cdots\}\] \[-n = [(0,n)] = \{(0,n),(1,n+1),\cdots\}\] and we consider $\mathbb{N}$ to be a subset of $Z$ when actually we have an injective map $\phi \colon \mathbb{N}\to \mathbb{Z}$ given by $\phi (n) = n = [(n,0)]$.
\end{exmp}

\begin{exmp}
We construct $\mathbb{Q}$ from $\mathbb{Z}$ as follows. we define $\sim$ on $\mathbb{Z}\times (\mathbb{Z}\times \{0\})$ by $(a,b)\sim (c,d) \Leftrightarrow a+d = b+c$. Then $\mathbb{Q} = (\mathbb{Z}\times (\mathbb{Z}\times \{0\}))/\sim$
\end{exmp}