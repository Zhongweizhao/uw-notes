\lecture{Sept. 13}

\topic{Women in Math}
Tue Sept. 13\\
4:30-6:00\\
DC 1301


\topic{ZFC Axioms}
\begin{itemize}
    \item Empty Set: there exist a set, denoted by $\emptyset$, with no elements.
    \item Equality: two sets are equal when they have the same elements. $A = B$ when for every set $x$, $x\in A \iff x\in B$
    \item Pair Axiom: if $A$ and $B$ are sets then so is $\{A,B\}$. In particular, taking $A=B$ shows that $\{A\}$ is a set.
    \item Union Axiom: if $S$ is a set of sets then $\cup_{S} = \bigcup_{A\in S} A= \{x\mid x\in A$ for some $A\in S\} $. If $A$ and $B$ are sets, then so is $\{A,B\}$ hence so is $A\cup B = \cup_{\{A,B\}}$
    \item Power Set Axiom: if $A$ is a set, then so is its Power Set $P(A)$. $P(A) = \{X\mid X\subseteq A\}$. In particular, $\emptyset \subseteq X$, $X\subseteq X$
    \item Axiom of Infinity: if we define
        \begin{align*}
            0 & = \emptyset \\
            1 & = \{0\} = \{\emptyset\} \\
            2 & = \{0,1\} = \{\emptyset,\{\emptyset\}\}\\
            3 & = \{0,1,2\} = \{\emptyset, \{\emptyset,\{\emptyset\}\}\}\\
            \vdots&\\
            n+1 & = n \cup \{n\}
        \end{align*}
        
        Then $\mathbb{N}=\{0,1,2,3,\dots\}$ is a set (called the set of natural numbers)
    \item Specification Axioms: if $A$ is a set, and $F(x)$ is a mathematical statement about an unknown set $x$, then $\{x\in A \mid F(x)\text{ is true }\}$ is a set.
    
    Examples:
    $$\{x\in\mathbb{N}\mid x\text{ is even }\} = \{0,2,4,6,\dots\}$$
    $$A \cap B = \{x\in A\cup B \mid x\in A \text{ and } x\in B\}$$
    \item Replacement Axioms: if $A$ is a set and $F(x,y)$ is a mathematical statement about unknown sets $x$ and $y$ with the property that for every $x\in A$ there is a unique set $y$ such that the statement is true, and if we denote this unique set $y$ by $y=F(x)$, then $\{F(x) \mid x\in A\}$ is a set.
    
    \item Axiom of Choice: if $S$ is a set of non-empty sets then there exists a function $F\colon S\to U_S$ which is called a choice function for $S$ such that $$F(A)\in A \quad \forall A\in S$$

\end{itemize}

\topic{Things that are sets}

\begin{exmp}
$$A \cup B = \{x\mid x\in A \text{ or }x\in B\}$$
$$A \cap B = \{x\in A\cup B \mid x\in A \text{ and } x\in B\}$$
$$A \backslash B = \{x\in A\mid x\notin B\}$$
$$A\times B = \{(x,y)\mid x\in A,x\in B\}$$
$$A^2=A\times A$$
\end{exmp}

\topic{One way to define ordered pairs}
$$(x,y) = \left\{\{x\},\{x,y\}\right\}$$
$$x\in A, y\in B \therefore x,y\in A\cup B$$
$$\{x\},\{x,y\}\in P(A\cup B)$$
$$(x,y) = \{\{x\},\{x,y\}\} \subseteq P(A\cup B)$$
$$(x,y) \in P(P(A\cup B))$$
$$\therefore A\times B = \{(x,y) \in P(P(A\cup B)) \mid x\in A \text{ and } y\in B\}$$


\topic{function}
When $A$ and $B$ are sets, a function from $A$ yo $B$ is a subset $F \subseteq A\times B$ with  the property that for every $x\in A$ there exists a unique $y\in B$ such that $(x,y)\in F$

When $F$ is a function from $A$ to $B$ we write $$F\colon A\to B$$

and for $x\in A$ and $y\in B$ we write $y=F(x)$ to indicate that $(x,y) \in F\subseteq A\times B$

\topic{Sequence}
A sequence $a_0,a_1,a_2,\dots$ of natural numbers is a function $a\colon \mathbb{N}\to\mathbb{N}$ and we write $a(k)$ as $a_k$

\topic{Less than}
the relation $<$ on $\mathbb{Z}$ is a subset $< \subseteq \mathbb{Z}^2$ and we write $x<y$ when $(x,y)\in <$


\topic{}
We can use the ZFC Axioms to define and construct 
\begin{itemize}

\item $\mathbb{Z}$: the set of integers
\item $\mathbb{Q}$: the set of rationals
\item $\mathbb{R}$: the set of real numbers 
\item $+,\times$: operations
\item $<,>$: relations
\end{itemize}









