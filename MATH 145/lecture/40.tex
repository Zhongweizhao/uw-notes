\lecture{Nov. 22}

\begin{exmp}
    Using RSA, Alice wants to receive a message from Bill. Alice chooses two large primes $p$, $q$, then calculate $n = pq$ and $\varphi = \varphi (n) = \varphi (p) \varphi (q) = (p-1)(q-1)$ (or $\varphi = lcm(p-1, q-1)$). Then she chooses $e$ with $gcd(e,\varphi) = 1$. Then calculate $d = e^{-1} \mod \varphi$. Then she makes public the pair $(n,e)$.
    
    Bob converts this message to a number $m$ with $m\leq n$ (or several such numbers). Bob calculates and sends $c = m^e \mod n$.
    
    Alice receives the message by calculating $c^d \mod n$. Since the table of powers repeats every $\varphi$ rows, and $ed = 1 \mod \varphi$, $c^d = {(m^e)}^d = m^{ed} = m^{1+k\varphi} = m \mod n$.
\end{exmp}

\topic{Here ends Chapter 4: Integers Modulo $n$}

\topic{Chapter 5: Complex numbers}

\[\mathbb{C} = \mathbb{R}^2 = \{(x,y) \mid x \in \mathbb{R}, y \in \mathbb{R} \}\]

In $\mathbb{C}$ we write \[0 = (0,0) , \ 1 = (1,0) , \ i = (0,1)\]

and for $x, y \in \mathbb{R}$ we write \[x = (x,0), iy = yi = (0, y)\]
\[x + iy = x+yi = (x,y).\]
For $z = x+ iy$ with $x, y \in \mathbb{R}$, x is called the real part of $z$ and $y$ is called the imaginary part and we write Re(z) = x, lm(z) = y.

We define addition and multiplication in $\mathbb{C}$ by
\[(a+ib)(c+id) = (a,b) + (c,d) = (a+c, b+d) = (a+c) + i(b+d)\]

and 
\[(a+ib)(c+id) = (ac-bd) + i(ad+bc) \]

\begin{thm}
$\mathbb{C}$ is a field.
\end{thm}

\begin{proof}
We only bother to show that every non-zero $z\in\mathbb{C}$ has a inverse.

Let $z = a + ib$ with $a,b\in\mathbb{R}$ and $(a,b)\neq (0,0)$. We need to find $w = x + iy$ with $x,y\in\mathbb{R}$ such that $zw = 1$.
That is \[(a+ib)(x+iy)= 1+0i\]
Then \[(ax-by)+i(ay+bx) = 1+0i\]
So we need \begin{gather*}
    ax-by = 1\\
    ay+bx = 0
\end{gather*}
by solving the equations we get \begin{gather*}
    x = \frac{a}{a^2+b^2}\\
    y = \frac{-b}{a^2+b^2}
\end{gather*}
Therefore \[w = x+iy = \frac{a}{a^2+b^2} + i\frac{-b}{a^2+b^2}\]
\end{proof}

\begin{defn}
For $z = x + iy$ with $x,y\in\mathbb{R}$, the conjugate of $z$ is \[\overline{z} = x -iy\] and the norm (or the length) of $z$ is \[\abs{z} = \sqrt{x^2+y^2}\]
\end{defn}

\begin{thm}[Properties of Conjugate and Norm]
    \begin{enumerate}
        \item $\overline{\overline{z}} = z$
        \item $\overline{z+w} = \overline{z}+\overline{w}$
        \item $\overline{z\cdot w} = \overline{z} \cdot \overline{w}$
        \item $z + \overline{z} = 2 \cdot Re(z)$
        \item $z - \overline{z} = 2 i \cdot Im(z)$
        \item $\abs{\overline{z}} = \abs{z}$
        \item $z \cdot \overline{z} = \abs{z}^2$
        \item If $z \neq 0$ then $\abs{z} \neq 0$ and $\displaystyle z^{-1} = \frac{\overline{z}}{\abs{z}^2}$
        \item $\abs{z} \geq 0$ with $\abs{z} = 0 \Leftrightarrow z = 0$
        \item $\abs{zw} = \abs{z}\abs{w}$ ($\abs{z+w} \neq \abs{z}+\abs{w}$)
        \item $\abs{\abs{z}+\abs{w}} \leq \abs{z+w} \leq \abs{z} + \abs{w}$ (Triangle Inequality)
    \end{enumerate}
\end{thm}

\begin{thm}
    Every non-zero complex number has exactly two complex square roots.
\end{thm}

\begin{proof}
Let $z = a+ib$ with $a,b\in\mathbb{R}$ and $(a,b)\neq (0,0)$. We need to find $w = x + iy$ with $x,y\in\mathbb{R}$ such that $w^2 = z$.

We have \begin{gather*}
    w^2 = z\\
    (x+iy)^2 = a+ib \\
    (x^2-y^2) + i(2xy) = a + ib\\
\end{gather*}

then \begin{gather*}
    x^2-y^2 = a\\
    2xy = b
\end{gather*}

by solving the equations with respect to $x$ we get 
\[x^2 = \frac{a + \sqrt{a^2+b^2}}{2}\]
\[y^2 = x^2-a = \frac{-a + \sqrt{a^2+b^2}}{2}\]

Then we must have \[w = x + iy = \pm \sqrt{\frac{a + \sqrt{a^2+b^2}}{2}} \pm i \sqrt{\frac{-a + \sqrt{a^2+b^2}}{2}} \]

To get $2xy = b$ we need that if $b>0$, \[w = x + iy = \pm \left( \sqrt{\frac{a + \sqrt{a^2+b^2}}{2}} + i \sqrt{\frac{-a + \sqrt{a^2+b^2}}{2}} \right)\]

and if $b < 0$, then \[w = x + iy = \pm \left( \sqrt{\frac{a + \sqrt{a^2+b^2}}{2}} - i \sqrt{\frac{-a + \sqrt{a^2+b^2}}{2}} \right)\]


and when $b = 0$, then \[w = \begin{cases} \pm \sqrt{a} & \text{if } a > 0 \\ \pm \sqrt{-a} & \text{if } a<0  \end{cases}\qedhere\]
\end{proof}

\begin{exmp}
    In $\mathbb{C}$, \begin{align*}
        \sqrt{3+4i} = & \pm \left( \sqrt{\frac{3 + \sqrt{3^2+4^2}}{2}} + i \sqrt{\frac{-3 + \sqrt{3^2+4^2}}{2}} \right)\\ =& \pm (2+i)
    \end{align*}
\end{exmp}







