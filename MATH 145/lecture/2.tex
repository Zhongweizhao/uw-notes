\lecture{Sept. 12}

\topic{Mathematics Contest}
\topic{Big Contests}
\begin{itemize}
\item Small C
\item Big E/Special K
\item Putnam
\item Bernoulli Trials  
\end{itemize}

\topic{Students Run}
\begin{itemize}
\item Integration Bee
\item over 6000
\end{itemize}

\topic{Others}
\begin{itemize}
\item Recreational Problem Sessions
\end{itemize}


\topic{ZFC Axioms}
\begin{itemize}
    \item Empty Set: there exist a set, denoted by $\emptyset$, with no elements.
    \item Equality: two sets are equal when they have the same elements. $A = B$ when for every set $x$, $x\in A \iff x\in B$
    \item Pair Axiom: if $A$ and $B$ are sets then so is $\{A,B\}$. In particular, taking $A=B$ shows that $\{A\}$ is a set.
    \item Union Axiom: if $S$ is a set of sets then $\cup_{S} = \bigcup_{A\in S} A= \{x\mid x\in A$ for some $A\in S\} $. If $A$ and $B$ are sets, then so is $\{A,B\}$ hence so is $A\cup B = \cup_{\{A,B\}}$
    \item Power Set Axiom: if $A$ is a set, then so is its Power Set $P(A)$. $P(A) = \{X\mid X\subseteq A\}$. In particular, $\emptyset \subseteq X$, $X\subseteq X$
    \item Axiom of Infinity: if we define
        \begin{align*}
            0 & = \emptyset \\
            1 & = \{0\} = \{\emptyset\} \\
            2 & = \{0,1\} = \{\emptyset,\{\emptyset\}\}\\
            3 & = \{0,1,2\} = \{\emptyset, \{\emptyset,\{\emptyset\}\}\}\\
            \vdots&\\
            n+1 & = n \cup \{n\}
        \end{align*}
        
        Then $\mathbb{N}=\{0,1,2,3,\dots\}$ is a set (called the set of natural numbers)
\end{itemize}

