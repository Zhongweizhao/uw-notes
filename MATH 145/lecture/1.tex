\lecture{Sept. 9}


\topic{Course Orientation and Organization}
\topic{About the Professor}
Stephen New\\
MC 5419\\
Ext 35554\\
Email: snew@uwaterloo.ca\\
Website: www.math.uwaterloo.ca/~snew\\
Office Hour: MTWF 11:30-12:30
\topic{Recommended Textbook}
\begin{itemize}
\item An Introduction to Mathematical Thinking by Will J. Gilbert, Scott A. Vanstone
\item Lecture Notes: Integers, Polynomials and Finite Fields by K. Davidson
\end{itemize}



\topic{Some Paradoxes}
There are lots of paradoxes in English, such as "This statement is false". 

There are also some paradoxes in Mathematical world.




\topic{Russell's Paradox} 
Let $X$ be the set of all sets. Let $S=\{A\in X| A\notin A\}$. Is $S \in S$?

\topic{Some Question}
To avoid such paradoxes, some question was raised.
\begin{enumerate}
    \item What is an allowable mathematical object?
	\item What is an allowable mathematical statement?
    \item What is an allowable mathematical proof?
\end{enumerate}

\topic{Mathematical Object}
Essentially all mathematical objects are (mathematical) sets. In math, a set is a certain specific kind of collection whose elements are sets. Not all collection of sets are called sets. For a collection to be a set, it must be constructable using specific rules. These rules are called the ZFC axioms of set theory (or the Zermelo–Fraenkel axioms along with the Axiom of Choice)

These axioms include (imply) the following:
\begin{itemize}
    \item Empty Set: there exist a set, denoted by $\emptyset$, with no elements.
    \item Equality: two sets are equal when they have the same elements. $A = B$ when for every set $x$, $x\in A \iff x\in B$
    \item Pair Axiom: if $A$ and $B$ are sets then so is $\{A,B\}$
    \item Union Axiom: if $S$ is a set of sets then $\cup_{S} = \{x|x\in A$ for some $A\in S\} $. If $A$ and $B$ are sets, then so is $\{A,B\}$ hence so is $A\cup B = \cup_{\{A,B\}}$
    
\end{itemize}

