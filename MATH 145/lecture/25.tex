\lecture{Oct. 26}

\begin{thm}[The Euclidean Algorithm with Back-substitution]
Let $a,b\in \mathbb{Z}$, and let $d = gcd(a,b)$. Then there exist $s,t\in\mathbb{Z}$ such that $as+bt = d$.
\end{thm}

The proof of the theorem provides an \textbf{Algorithm} (that is a systematic procedure) called the \textbf{The Euclidean Algorithm} for computing $d = gcd(a,b)$ and an algorithm, called \textbf{Back-Substitution}, for finding $s,t\in\mathbb{Z}$ such that $as+bt = d$.

\begin{proof}
If $b\mid a$, then $gcd(a,b) = \abs{b}$ and we can take $s=0$ and $t = \pm 1$ to get $as+bt = d$.

Suppose $b\nmid a$.


Then apply the Division Algorithm repeatedly to get 
\[a = q_1b+r_1\]
\[b = q_2r_1 + r_2\]
\[r_1 = q_3 r_2 + r_3\]
\[\cdots\]
\[r_{n-3} = q_{n-1}r_{n-2} + r_{n-1}\]
\[r_{n-2} = q_{n}r_{n-1} + r_n\]
\[r_{n-1} = q_{n+1}r_n+0\]

\[gcd(a,b) = gcd(b,r_1) = gcd(r_1,r_2) = \cdots = gcd(r_n, 0) = r_n\]
Thus $d = gcd(a,b) = r_n$, the last non-zero remainder.

We have \begin{align*}
    d = r_n = & r_{n-2} - q_nr_{n-1} \\
    = & S_0r_{n-2} + S_1 (r_{n-3} - q_{n-1}r_{n-2})  \textit{ where } S_0 = 1 \textit{ and } S_1 = -q_n\\
    = & S_1 r_{n-3} + (S_0 - q_{n-1}S_1)r_{n-2} \\
    = & S_1 r_{n-3} + S_2 r_{n-2}  \textit{ where } S_2 = S_0 - q_{n-1}S_1 \\
\end{align*} 

We have a sequence $(S_l)_{l\geq 0}$ by $S_0 = 1$, $S_1 = -q_n$ and \[S_{l+1} = S_{l-1} - q_{n-l}S_l\]

We claim that \[d= r_k = S_{l-1}r_{n-l-1} + S_lr_{n-l}.\] 
Proof by induction: $\cdots.$
\end{proof}



\begin{exmp}
Let $a = 5151$ and $b = 1632$. Find $d = gcd(a,b)$ and find $s,t\in\mathbb{Z}$ such that $as+bt = d$.
\end{exmp}

\begin{solution}
\[5151 = 1632 \cdot 3(q_1) + 255\]
\[1632 = 255 \cdot 6(q_2) + 102\]
\[255 = 102 \cdot 2(q_3) + 51\]
\[102 = 51 \cdot 2 + 0\]
Thus $d = gcd(a,b) = 51$

\[S_0 = 1\]
\[S_1 = -q_3 = -2\]
\[S_2 = S_0 - S_1q_2 = 13\]
\[S_3 = S_1 - S_2q_1 = -41\]

So we can take $s= 13$ and $t = -41$ to get $as+bt = d$
\end{solution}


\begin{exmp}
Let $a=754$ and $b=-3973$. Find $d = gcd(a,b)$ and find $s,t\in\mathbb{Z}$ such that $as+bt = d$.
\end{exmp}
\begin{solution}
\[3973 = 754 \cdot 5(q_1) + 203\]
\[754 = 203 \cdot 3(q_2) + 145\]
\[203 = 145 \cdot 1(q_3) + 58\]
\[145 = 58 \cdot 2(q_4) + 29\]
\[58 = 29 \cdot 2 + 0\]

Thus $d = gcd(a,b) = 29$

\[S_0 = 1\]
\[S_1 = -q_4 = -2\]
\[S_2 = S_0 - S_1q_3 = 3\]
\[S_3 = S_1 - S_2q_2 = -11\]
\[S_4 = S_2 - S_3q_1 = 58\]

Thus $(3973)(-11) + (754)(58) = 29$

Thus we can take $s = 58$ and $t = 11$ to get $as + bt = d$
\end{solution}


\begin{thm}[More Properties of GCD]
Let $a,b,c\in\mathbb{Z}$
\begin{enumerate}
\item if $c\mid a $ and $c\mid b$ then $c\mid gcd(a,b)$
\item there exist $x,y\in\mathbb{Z}$ such that $ax+by = c$ iff $gcd(a,b)\mid c$
\item there exist $x,y\in\mathbb{Z}$ such that $ax+by = 1$ iff $gcd(a,b) = 1$
\item if $d = gcd(a,b) \neq 0$ (which is the case unless $a=b=0$) then $gcd(a/d,b/d) = 1$
\item if $a\mid bc$ and $gcd(a,b) = 1$ then $a\mid c$
\end{enumerate}
\end{thm}

\begin{proof}
\begin{enumerate}
\item[5.] Let $a,b,c\in\mathbb{Z}$. Suppose $a\mid bc$ and $gcd(a,b) = 1$. Since $a\mid bc$, choose $k\in\mathbb{Z}$ such that $bc = ak$. Since $gcd(a,b) = 1$, we can choose $s,t\in\mathbb{Z}$ such that $as+bt = 1$. Then $c=c\cdot 1 = c\cdot (as+bt) = acs + bct = acs + akt = a(cs + kt)$. So $a\mid c$
\end{enumerate}
\end{proof}