\lecture{Oct. 31}

\begin{note}
There exist arbitrary large gaps between prime numbers. 
\end{note}

\begin{thm}[Bertrand's postulate]
For every $n\in\mathbb{Z}^+$ there is a prime $p$ with $n< p \leq 2n$
\end{thm}

\begin{thm}[Dirichlet's Theorem on Primes in Arithmetic Progression]
Let $a,b\in\mathbb{Z}^+$ with $gcd(a,b) = 1$. Then there exists infinitely many primes $p$ of the form $p=a+tb$ for some $t\in\mathbb{Z}$. In other words, there exist infinitely many primes in the sequence \[a,a+b,a+2b,a+3b,\cdots\]
\end{thm}

\begin{thm}[The Prime Number Theorem]
    For $x\in\mathbb{R}$ let $\pi (x)$ denote the number of primes $p$ with $p\leq x$. Then \[\pi (x) ~ \frac{x}{lnx}\] which means that \[\lim_{x\to\infty}\frac{\pi (x)}{x/lnx} = 1\]
\end{thm}

\begin{conj}[$n^2$ Conjecture]
For all $n\in\mathbb{Z}^+$ there exists a prime p with $n^2<p < (n+1)^2$
\end{conj}

\begin{conj}[$n^2+1$ Conjecture]
There are infinitely many primes of the form $p=n^2+1$ for some $n\in\mathbb{Z}$
\end{conj}

\begin{conj}[Mersenne Primes Conjecture]
    There exist infinitely many primes of the form $p = 2^n-1$ for some $n\in\mathbb{Z}^+$ (such primes are called Mersenne Primes).
\end{conj}

\begin{exer}
If $2^n-1$ is prime, then $n$ is prime.
\end{exer}

\begin{conj}[Fermat Primes Conjecture]
    There are only finitely many primes of the form $p=2^n+1$ with $n\in\mathbb{Z}^+$ (such primes are called Fermat primes).
\end{conj}

\begin{exer}
If $2^n+1$ is prime then $n=2^k$ for some $k\in\mathbb{N}$
\end{exer}

\begin{conj}[Twin Primes Conjecture]
    There exist infinitely many primes $p$ such that $p+2$ is also prime. Such primes $p$ and $p+2$ are called twin primes.
\end{conj}

\begin{conj}[Goldbach's Conjecture]
    Every even number $n\geq 2$ is a sum of two primes.
\end{conj}


\begin{thm}[Unique Prime Factorization]
    Every integer $n\geq 2$ can be expressed uniquely in the form \[n=\prod_{i=1}^l p_i = p_1p_2\cdots p_l\] for some $l\in\mathbb{Z}^+$ and some primes $p_1,p_2,\cdots,p_l$ with $p_1\leq p_2\leq \cdots \leq p_l$.
\end{thm}

\begin{proof}
    First we show existence. Let $n\geq 2$. Suppose, inductively, that every integer $k$ with $2\leq k < n$ can be written (uniquely) in the required form. If $n$ is prime then $n = p_1$ with $p_1 = n$.
    
    Suppose $n$ is composite, say $n=ab$ with $1<a<n$ and $1<b<n$. Since $2\leq a < n$ and $2\leq n < n$ we can write \[a = \prod_{i=1}^l p_i\] and \[b = \prod_{j=1}^m q_j\] with $l,m\in\mathbb{Z}$ and the $q_j,p_i$ are primes.
    
    Thus \begin{align*}
        n = & ab \\
        = & p_1p_2\cdots p_lq_1q_2\cdots q_m\\
        = & r_1r_2\cdots r_{l+m}
    \end{align*}
    
    where the $(l+m)-$tuple $(r_1,r_2,\cdots ,r_{l+m})$ is obtained by rearranging the entries of the \[\text{(l+m)-tuple } (p_1,p_2,\cdots ,p_{l},q_1,q_2,\cdots ,q_{m})\] into non-decreasing order.
    
    Next we prove uniqueness. We need to show that if $n=p_1p_2\cdots p_l$ and $n=q_1q_2\cdots q_m$ where $l,m\in\mathbb{Z}^+$ and the $p_i$ and $q_j$ are primes with $p_1\leq p_2 \leq \cdots \leq p_l$ and $q_1\leq q_2\leq \cdots q_m$, then $l=m$ and $p_i=q_i$ for all $i$.
    
    Suppose $n=p_1p_2\cdots p_l=q_1q_2\cdots q_m$ as above. Since $n=p_1p_2\cdots p_l$ we have $p_1\mid n$. Since $n=q_1q_2\cdots q_m$ we have $p_1\mid q_1q_2\cdots q_m$. It follows that $p_1\mid q_k$ for some $k$ with $1\leq k\leq m$. Say $p_1\mid q_k$. Since $q_k$ is prime, its only positive divisors are $1$ and $q_k$. Since $p_1\neq 1$, so $p_1 = q_k$. Similarly, $q_1 = p_j$ for some $j$ with $1\leq j\leq l$. Since $p_1 = q_k \geq q_1 = p_j \geq p_1$, so we must have $p_1=p_j=q_1$.
\end{proof}