\lecture{Nov. 30}

\begin{defn}
A unique factorization domain (or UFD) is an integral domain $R$ such that 
\begin{enumerate}
    \item for every non-zero non-unit $a\in R$ we have $a = p_1p_2\cdots p_l$ for some $l\in\mathbb{Z}^+$ and some irreducible elements $p_i$ and
    \item the above factorization is unique up to order and association: for every non-zero non-unit $a\in R$, if $a = p_1p_2\cdots p_l = q_1q_2\cdots q_m$ with $l,m\in\mathbb{Z}^+$ and the $p_i,q_j$ are all irreducible, then $l=m$ and there exists a bijection \[\sigma \colon \{1,2,\cdots l\} \to \{1,2,\cdots l\}\] such that $p_i \sim q_{\sigma (i)}$ for all $i\in \{1,2,\cdots l\}$
\end{enumerate}
\end{defn}

\begin{note}
If in an integral domain R, we have a function $N\colon R\to \mathbb{N}$ such that 
\begin{enumerate}
\item $N(a) = 0 \Leftrightarrow a = 0$
\item $N(a) = 1 \Leftrightarrow a$ is a unit.
\item for all non-zero non-unit $a,b,c\in\mathbb{R}$, if $a = bc$, then $N(b) < N(a)$ and $N(c) < N(a)$.
\end{enumerate}
then (by induction) property (1) holds in the definition of a UFD. 
\begin{enumerate}
\item[4] for all $a,b\in R$ with $b\neq 0$ there exist $q,r \in R$ such that $a = qb + r$ and $N(r) < N(B)$.
\end{enumerate}
then as in $Z$ we can use the Euclidean Algorithm with Back-Substitution to find $d = gcd(a,b)$ and find $s,t\in R$ such that $as+bt = d$ and to show that every irreducible $a \in R$ is also prime and hence to prove that $R$ is a UFD
\end{note}


\begin{rem}[Remark About Polynomials]\leavevmode

\begin{enumerate}
\item Given a polynomial $f(x) = \sum_{i=0}^n {c_i}^{x_i}\in R[x]$ with $c_i \in R$ with $c_n \neq 0$ so $deg(f) = n$, then we have a corresponding function $f\colon R\to R$ given by $f(x) = \sum_{i=0}^n {c_i}^{x_i}$.
\item If $R$ is not commutative, then the product of polynomials is not the same as the product of their function. 
\item When $R$ is finite, equality in $R[x]$ is not the same as equality in $R^R$.
\item if $f$ is an integral domain, then $deg(fg) = deg(f) + deg(g)$.q
\end{enumerate}
\end{rem}