\lecture{Oct. 25}

\begin{exmp}
\begin{align*}
    (1+i)^10 = & (\sqrt{2} e^{i\pi /4})^{10} \\
    = & 32 e^{i5\pi /2}\\
    = & 32 e^{i\pi /2}\\
    = & 32 i
\end{align*}
\end{exmp}

\begin{note}[Non-uniqueness of the Polar Representation]
For $r,s,\alpha,\beta \in \mathbb{R}$ with $r,s>0$,
\[re^{i\alpha} = se^{i\beta} \Longleftrightarrow r=s \text{ and } \alpha = b \bmod 2\pi\]
\end{note}

\begin{thm}
Every non-zero complex number has exactly $n$ distinct complex n-th roots for $n \in \mathbb{Z}^+$. For $z = re^{i\theta}$ with $r>0$, the n-th roots of $z$ are the complex numbers \[w = \sqrt[n]{r} e^{i(\theta + 2\pi k) / n}\] with $k\in\mathbb{Z}_n$
\end{thm}

\begin{proof}
Let $z = re^{i\theta}$ with $r,\theta \in \mathbb{R}$ with $r>0$. We need to solve $w^n = z$ for $w\in\mathbb{C}$. Let $w = se^{i\alpha}$ with $s,\alpha \in \mathbb{R}$, $s > 0$. Then
\begin{align*}
    w^n = z \Longleftrightarrow & (se^{i\alpha})^n = re^{i\theta} \\
    \Longleftrightarrow & s^ne^{in\theta} = re^{i\theta} \\
    \Longleftrightarrow & s^n = r \text{ and } n\alpha = \theta + 2\pi k \text{ for some } k\in\mathbb{Z}\\
    \Longleftrightarrow & s = \sqrt[n]{r} \text{ and } \alpha = \frac{\theta }{n} + \frac{2\pi k}{n}\text{ for some } k\in\mathbb{Z}\\
    \Longleftrightarrow & w = se^{i\alpha} = \sqrt[n]{r}e^{i(\theta + 2\pi k) / n} \text{ for some }k\in \mathbb{Z} \\
    \Longleftrightarrow & w = se^{i\alpha} = \sqrt[n]{r}e^{i(\theta + 2\pi k) / n} \text{ for some }k\in \mathbb{Z}_n \\
\end{align*}
\end{proof}

\begin{nota}
When $x$ is real (and non negative), $\sqrt{x} = x^{1/2}$ normally denotes the unique non-negative square root of $x$. When $z$ is complex with $z\neq 0$, $\sqrt{z}$ sometimes denotes one of the two square roots, and sometimes denotes both. Similar remarks hold for n-th roots of $z$.
\end{nota}

\begin{exmp}
Find \[\sqrt[6]{-1 + \sqrt{3}i}\]
\end{exmp}

\begin{solution}
\begin{gather*}
    -1 + \sqrt{3} i = 2 e^{i 2\pi /3}\\
    \sqrt[6]{-1 + \sqrt{3}i} = \sqrt[6]{2}e^{i(\pi/9 + k\pi / 3)} \text{ for } k \in \mathbb{Z}_6
\end{gather*}
\end{solution}

\begin{exmp}[Application]
Find a closed-form formula for $x_n$ where $x_0 = 1$, $x_1 = 1$, $x_n = 2x_{n-1} - 5x_{n-2}$ for $n\geq 2$.
\end{exmp}

\begin{solution}
Let $f(z) = z^2 -2z + 5$. The roots for $f(z)$ are \[z = 1 \pm 2 i\]

Then by the Linear Recursion Theorem, there exist $A,B \in \mathbb{C}$ such that 
\[x_n = A(1+2i)^n + B(1-2i)^n\]
To get $x_0 = 1$ we have $A+B = 1$.

To get $x_1 = 1$ we have $(A+B) + 2i(A-B) = 1$.

Then $A = B = 1/2$. Therefore \[x_n = \frac{1}{2}(1+2i)^n + \frac{1}{2}(1-2i)^n\]

Note that $1 + 2i = \sqrt{5} e^{i \theta}$ with $\theta = \tan ^{-1} 2$. So 
\begin{align*}
    x_n = & \frac{1}{2}(1+2i)^n + \frac{1}{2}(1-2i)^n \\
    =& \frac{1}{2}(\sqrt{5} e^{i \theta})^n + \frac{1}{2}(\sqrt{5} e^{-i \theta})^n \\
    = & \frac{\sqrt{5}^n}{2} (e^{in\theta} + e^{-in\theta})\\
    = & \frac{\sqrt{5}^n}{2} (\cos (n\theta) + i\sin (n\theta ) + \cos (n\theta) - i\sin (n\theta ))\\
    = & \sqrt{5}^n \cos (n\theta )
\end{align*}
Thus \[x_n = \sqrt{5}^n \cos (n\tan^{-1} 2)\]
\end{solution}

\begin{exmp}
Find \[\sum_{k=0}^{\infty} \mybinom{l}{1+3k}\] where $\mybinom{m}{l} = 0$ for $l>m$.
\end{exmp}

\begin{solution}
Let $\alpha = e^{i 3\pi /3}$, then $1 + \alpha + \alpha^2 =0$.

\begin{align*}
    (1 + 1)^m = & \mybinom{n}{0} + \mybinom{n}{0} + \mybinom{n}{0} + \mybinom{n}{0} + \mybinom{n}{0} + \cdots \\
    \alpha ^2 (1+\alpha)^m = &  \mybinom{n}{0} \alpha^2 + \mybinom{n}{0} + \mybinom{n}{0} \alpha + \mybinom{n}{0}\alpha^2 + \mybinom{n}{0}  + \cdots \\
    \alpha ^2 (1+\alpha^2)^m = &  \mybinom{n}{0} \alpha + \mybinom{n}{0} + \mybinom{n}{0} \alpha^2 + \mybinom{n}{0}\alpha + \mybinom{n}{0}  + \cdots \\
\end{align*}

\begin{align*}
    3 \sum_{k=0}^{\infty} \mybinom{l}{1+3k} = & (1 + 1)^m + \alpha ^2 (1+\alpha)^m +\alpha ^2 (1+\alpha^2)^m  \\
    = &2^m + 2\cos \frac{(m-2)\pi}{3}
\end{align*}
\end{solution}

