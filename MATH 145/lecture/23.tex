\lecture{Oct. 24}
Woman in Pure Math/Math Finance Lunch

Tuesday 12:30-1:20 MC5417

\begin{thm}
\leavevmode
\begin{enumerate}
    \item if $b\neq 0$ and $a\mid b$ then $\abs{a} \leq \abs{b}$
    \item $a\mid a$
    \item if $a\mid b$ and $b\mid a$ then $a=b$
    \item if $a\mid b$ and $b\mid c$ then $a\mid c$
    \item if $a\mid b$ and $a\mid c$ then \[\forall x,y\in \mathbb{Z} \ a\mid (bx+cy)\]
\end{enumerate}
\end{thm}

\begin{proof}
\leavevmode
\begin{enumerate}
    \item Let $a,b\in\mathbb{Z}$. Suppose $b\neq 0$ and $a\mid b$. Since $a\mid b$ we can choose $k\in\mathbb{Z}$ so that $b=ak$. Note that $k\neq 0$ because if $k=0$ then $b = 0$ but $b\neq 0$. Since $k\neq 0$ we have $\abs{k}\geq 1$. So we have 
    \begin{align*}
        b =& ak \\
        \abs{b} = & \abs{ak} \\
         = & \abs{a} \abs{k}\\
        \geq & \abs{a}\cdot 1 \\
        = & \abs{a}
    \end{align*}
    \item Let $a\in\mathbb{Z}$. Since $a= a\cdot 1$, it follows that $a\mid a$.
    \begin{align*}
        \{\forall x \ x\cdot 1 = x\} \vDash & \forall x \ x\cdot 1 = x \\
        \vDash & a\cdot 1 = a \\
        \vDash & \exists x \ a\cdot x = a
    \end{align*}
    \item Let $a,b\in\mathbb{Z}$. Suppose $a\mid b$ and $b\mid a$. Choose $k\in\mathbb{Z}$ so that $b=ak$. Choose $l\in\mathbb{Z}$ sp that $a=bl$. Then $b=ak=(nl)k = b(lk)$
    \begin{align*}
        b - b(lk) =& 0 \\
        b\cdot 1 - b(lk) = & 0 \\
        b(1-lk) = & 0 
    \end{align*}
    So $b=0$ or $(1-lk) = 0$ (Since $\mathbb{Z}$ has no zero divisors.)
    
    Case 1: Suppose $b=0$, then $a = bl = 0\cdot l = 0$, so we have $b = a = 0$, hence $b=\pm a$.
    
    Case 2: Suppose $1-lk = 0$, then $lk = 1$ and so either $l=k=1$ or $l=k=-1$. When $l=k=1$, we have $b = ak = a\cdot 1 = a$, then $b =\pm a$. When $l=k=-1$, we have $b = ak = a (-1) = (-1) a = -a$, then $b=\pm a$.
    
    In all cases we have $b=\pm a$ as required.
    
    \item $cdots$
    \item Let $a,b,c\in\mathbb{Z}$. Suppose $a\mid b$ and $a\mid c$. Say $b=ak$ and $c=al$ with $k,l\in\mathbb{Z}$. Let $x,y\in\mathbb{Z}$.
    \begin{align*}
        bx + cy =& (ak)x + (al)y \\ 
        = & a(kx) + a(ly) \\
        =& a(kx+ly)
    \end{align*}
    $\therefore a\mid bx+cy$ as required.
\end{enumerate}
\end{proof}

\begin{rem}
$a\mid b$ means $\exists x \ b = ax$. $a\mid c$ means $\exists x \ c = ax$.

\begin{align*}
    &[\exists x \ b = ax]_{b\mapsto bx+cy}\\
    \equiv&[\exists u \ b = au]_{b\mapsto bx+cy}\\
    \equiv&\exists u \ (bx+cy)= au
\end{align*}
$a\mid (bx+cy)$ means $\exists u \ (bx+cy)= au$
\end{rem}

\begin{rem}
    Recall that when $b\neq 0$, if $a\mid b$ then $\abs{a} \leq \abs{b}$. So $b$ has finitely many divisors (and the greatest divisor is $\abs{b}$).
\end{rem}

\begin{defn}
    For $a,b,d\in\mathbb{Z}$, we say that $d$ is a \textbf{common divisor} of $a$ and $b$ when $d\mid a$ and $d\mid b$. When $a$ and $b$ are not both zero, there are only finitely many common divisor of $a$ and $b$, and $\pm 1$ are common divisors, so $a$ and $b$ do have a greatest common divisor and we denote it by $gcd(a,b)$.
    
    For convenience, we also write $gcd(0,0) = 0$
\end{defn}

\begin{thm}\textbf{(Properties of the GCD)}
Let $a,b,c\in\mathbb{Z}$.
\begin{enumerate}
\item $gcd(a,b) = gcd(b,a)$
\item $gcd(a,b) = gcd(\abs{a},\abs{b})$
\item if $a\mid b$ then $gcd(a,b) = \abs{a}$, in particular, $gcd(a,0) = \abs{a}$
\item $gcd(a,b) = gcd(a+tb,b)$ for all $t\in\mathbb{Z}$.
\item if $a=qb + r$ where $q,r\in\mathbb{Z}$, then $gcd(a,b) = gcd(b,r)$
\end{enumerate}
\end{thm}

\begin{proof}
\begin{enumerate}
\item[4] To show that $gcd(a,b) = gcd(a+tb,b)$ we shall show that the common divisor of $a$ and $b$ is exactly the same as the common divisor of $a+tb$ and $b$. 

Let $a,b,t\in\mathbb{Z}$. Let $d\in\mathbb{Z}$. Suppose $d\mid a$ and $d\mid b$ then $d\mid ax+by$ for all $x,y\in\mathbb{Z}$. In particular, $d\mid (a\cdot 1+bt)$, so $d\mid (a+td)$. Thus $d\mid (a+tb)$ and $d\mid b$.

Conversely, suppose $d\mid (a+tb)$ and $d\mid b$. Then $d\mid (a+tb) x + by$  for all $x,y\in\mathbb{Z}$. In particular, $d\mid (a+tb)\cdot 1 + b\cdot (-1) $, so $d\mid a$. Thus $d\mid a$ and $d\mid a$. 
\end{enumerate}
\end{proof}