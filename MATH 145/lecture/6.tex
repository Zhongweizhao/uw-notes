\lecture{Sept. 19}

\topic{Tautology}

Let $F$ and $G$ be formula and let $S$ be a set of formulas 

\begin{nota}
We say that $F$ is a tautology, and we write $\vDash F$, when $\alpha (F) =1$ for every assignment $\alpha$



We say that F is a contradiction when $\alpha (F) = 0$ for every assignment $\alpha$, or equivalently when $\vDash \neg F$

We say that $F$ is equivalent to $G$, and we write $F\equiv G$ when $\alpha (F) = \alpha (G)$ for every assignment $\alpha$

We say that argument "S therefore G" is valid, or that "S induces G" or that "G is a consequence of S", when for every assignment $\alpha$ for which $\alpha (F) = 1$ for every $F\in S$ we have $\alpha (G) = 1$. 

When $S = \{F_1, F_2,\dots ,F_n\}$ we have $S\vDash G$ is equivalent to $\{((F_1\wedge F_2)\wedge \dots \wedge F_n)\} \vDash G$ which is equivalent to $\vDash (((F_1\wedge F_2)\wedge \dots \wedge F_n)\to G)$
\end{nota}

When we consider an argument "$S$ therefore $G$", the formula in $S$ are called the premises for the hypothesis or the assumption and the formula $G$ is called the conclusion of the argument.

Here are some examples of tautology.

\begin{enumerate}
\item $\vDash F \vee \neg F$
\item $\vDash P \to P$
\item $\vDash P \leftrightarrow P$
\item $\vDash \neg (P\wedge \neg P)$
\item $\vDash \neg P \to (P \to Q)$
\item $\vDash Q\to (P\to Q)$
\end{enumerate}

Here are some truth equivalences

\begin{enumerate}
\item $P\equiv P$
\item $P\equiv \neg \neg P$
\item $P\vee Q \equiv Q \vee P$
\item $P\wedge Q \equiv Q\wedge P $
\item $P\leftrightarrow Q \equiv Q\leftrightarrow P$
\item $(P\wedge Q) \wedge R \equiv P\wedge (Q\wedge R)$
\item $(P\vee Q) \vee R \equiv P\vee (Q\vee R)$
\item $P\wedge (Q \vee R) \equiv (P\wedge Q)\vee (P \wedge R)$
\item $P\vee (Q \wedge R) \equiv (P\vee Q)\wedge (P \vee R)$
\end{enumerate}

Here are some valid argument 

\begin{enumerate}
\item $\{P\} \vDash P$
\item $\{P\wedge Q \} \vDash P$
\item $\{P\wedge Q \} \vDash Q$
\item $P \vDash \{P\wedge Q \}$
\item $Q \vDash \{P\wedge Q \}$
\item $\{\neg P\}\vDash P\to Q$
\item $\{Q\}\vDash P\to Q$
\item $\{P,Q\}\vDash P\leftrightarrow Q$
\item $\{\neg P,\neg Q\}\vDash P\leftrightarrow Q$
\item $\{P, P\to Q\}\vDash Q$
\end{enumerate}


\begin{exmp}
Determine whether $$\vDash (P\to (Q\to R))\to ((P\to Q) \to (P\to R))$$
\end{exmp}

\begin{solution}
We make a truth table for $$F = (P\to (Q\to R))\to ((P\to Q) \to (P\to R))$$

\begin{center}
\begin{tabular}{c|c|c|c|c}
     $P$ & $Q$ & $R$ & $(P\to (Q\to R))$ &  $((P\to Q) \to (P\to R))$\\\hline
     1 & 1 & 1 & 1 & 1 \\\hline
     1 & 1 & 0 & 0 & 0 \\\hline
     1 & 0 & 1 & 1 & 1 \\\hline
     1 & 0 & 0 & 1 & 1 \\\hline
     0 & 1 & 1 & 1 & 1 \\\hline
     0 & 1 & 0 & 1 & 1 \\\hline
     0 & 0 & 1 & 1 & 1 \\\hline
     0 & 0 & 0 & 1 & 1 
     
\end{tabular}
\end{center}
\end{solution}

Here are some relationships between tautologies, equivalences and validity.

\begin{align*}
    F\equiv G &\Leftrightarrow  \vDash (F\leftrightarrow G)\\
    &\Leftrightarrow  \vDash ((F\to G)\wedge (G\to F)) \\
    &\Leftrightarrow  \{F\} \vDash G \text{ and }\{G\} \vDash F \\
\end{align*}


When $S = \{F_1,F_2,\dots , F_n\} $,
\begin{align*}
S \vDash G &\Leftrightarrow \{F_1,F_2,\dots , F_n\} \vDash G\\
&\Leftrightarrow (((F_1\wedge F_2)\wedge \dots \wedge F_n) \vDash G\\
&\Leftrightarrow \vDash (((F_1\wedge F_2)\wedge \dots \wedge F_n) \to G\\
\end{align*}

Also $$\vDash F \Leftrightarrow \emptyset \vDash F$$

$\vDash F$ means for all assignment $\alpha$, $\alpha (F) = 1 $

$\emptyset \vDash F$ means for all assignment $\alpha$, if (for every $G \in \emptyset, \alpha (G) = 1$ ) then $\alpha (F) = 1$


\begin{nota}
For a set $A$ and a statement or formula $F$
$$\forall x\in A \enspace F \text{ means } \forall x (x\in A \to F)$$
and 
$$\exists x\in A \enspace F \text{ means } \forall x (x\in A \wedge F)$$
\end{nota}

So (for every $G \in \emptyset, \alpha (G) = 1$ ) is always true. This proves that $\vDash F \Leftrightarrow \emptyset \vDash F$.

\begin{exmp}
 Determine whether $$(P \vee Q)\to R \equiv (P\to R)\wedge (Q\to R)$$
\end{exmp}
\begin{solution}
\begin{center}
\begin{tabular}{c|c|c|c|c}
     $P$ & $Q$ & $R$ & $(P \vee Q)\to R$ &  $(P\to R)\wedge (Q\to R)$\\\hline
     1 & 1 & 1 & 1 & 1 \\\hline
     1 & 1 & 0 & 0 & 0 \\\hline
     1 & 0 & 1 & 1 & 1 \\\hline
     1 & 0 & 0 & 0 & 0 \\\hline
     0 & 1 & 1 & 1 & 1 \\\hline
     0 & 1 & 0 & 0 & 0 \\\hline
     0 & 0 & 1 & 1 & 1 \\\hline
     0 & 0 & 0 & 1 & 1 
\end{tabular}
\end{center}

\end{solution}

\begin{exmp}
Determine whether $$\{P\to (Q\vee \neg R),Q\to \neg P\}\vDash R\to \neg P$$
\end{exmp} 

\begin{solution}

\begin{center}
\begin{tabular}{c|c|c|c|c|c|c|c|c}
    $P$ & $Q$ & $R$ & $\neg R$ & $Q\vee\neg R$ & $\neg P$ & $P\to (Q\vee\neg R)$ & $Q\to\neg P$ & $R\to\neg P$\\ \hline
    1 & 1 & 1 & 0 & 1 & 0 & 1 & 0 & 0 \\\hline
    1 & 1 & 0 & 1 & 1 & 0 & 1 & 0 & 1 \\\hline
    1 & 0 & 1 & 0 & 0 & 0 & 0 & 1 & 0 \\\hline
    1 & 0 & 0 & 1 & 1 & 0 & 1 & 1 & 1 \\\hline
    0 & 1 & 1 & 0 & 1 & 1 & 1 & 1 & 1 \\\hline
    0 & 1 & 0 & 1 & 1 & 1 & 1 & 1 & 1 \\\hline
    0 & 0 & 1 & 0 & 0 & 1 & 1 & 1 & 1 \\\hline
    0 & 0 & 0 & 1 & 1 & 1 & 1 & 1 & 1 \\
\end{tabular}
\end{center}
\end{solution}





