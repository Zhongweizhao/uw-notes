\lecture{Nov. 11}
\begin{thm}[Fermat's Little Theorem]
let $p$ be a prime then
    \begin{enumerate}
        \item for all $a\in \mathbb{Z}$ such that $gcd(a,p) = 1$, \[a^{p-1} = 1 \mod p\]
        \item for all $a\in \mathbb{Z}$, \[a^p  = a \mod p\]
    \end{enumerate}
\end{thm}

\begin{thm}[Euler-Fermat Theorem]
Let $n \in \mathbb{Z}^+$. For all $a\in\mathbb{Z}$ with $gcd(a,n) = 1$, \[a^{\varphi (n)} = 1 \mod n\]
\end{thm}

\begin{exmp}
Find $2^{-1}$ in $\mathbb{Z}_11$
\end{exmp}

\begin{solution}[Solution 1]
In $\mathbb{Z}_11$, $2^{-1} = 6$ because $2\cdot 6 = 12 = 1$.
\end{solution}

\begin{solution}[Solution 2]
Since $2^{10} = 1 \mod 11$ by Fermat's Little Theorem, so $2^{-1} = 2^{9} = 6 \mod 11$.
\end{solution}

\begin{defn}[Cyclic]
We say that a group $G$ with $\abs{G} = n$ is cyclic and is generated by $u\in G$ when \[G = <u> = \{u^k \mid  k\in\mathbb{Z}\}\]
\end{defn}

\textbf{Fact: }When $p$ is an odd prime, $U_p^{k}$ is cyclic.

\begin{rem}
\[U_11 = <2> = <2^k> \text{ for all }k\in U_{10} =  <2>= <5> = <7> = <6>\]
\end{rem}

\begin{exmp}
Consider the Diophantine equation $x^2+y^2 = n$ where $n\in\mathbb{N}$. Show that if $n=3 \mod 4$ then there are no solutions.
\end{exmp}
\begin{solution}
In $\mathbb{Z}_4$, \[
\begin{array}{ccccc}
    x & 0 & 1 & 2 & 3 \\
    x^2 & 0 & 1 & 0 & 1 \\
\end{array}\]

For $x,y\in\mathbb{Z}_4$,
\begin{align*}
    x^2+y^2 \in & \{0+0,0+1,1+0,1+1\}\\
    = & \{0,1,2\}
\end{align*}
\end{solution}


\begin{solution}
In $\mathbb{Z}_7$, \[
\begin{array}{cccccccc}
    x & 0 & 1 & 2 & 3 & 4 & 5 & 6 \\
    x^2 & 0 & 1 & 4 & 2 & 2 & 4 & 1 \\
    x^3 & 0 & 1 & 1 & 6 & 1 & 6 & 6 \\
    3x^2 & 0 & 3 & 5 & 6 & 6 & 5 & 3 \\
    3x^2+4 & 4 & 0 & 2 & 3 & 3 & 2 & 0
\end{array}\]

For $x,y\in\mathbb{Z}_7$, since $3x^2 +4 = y^3$ in $\mathbb{Z}_7$, \

It follows that if $3x^2 +4 = y^3$ in $\mathbb{Z}_7$, then $x=0,6 \mod 7$ and $y=0 \mod 7$.
\end{solution}

\begin{exer}
Try the example in $\mathbb{Z}_9$.
\end{exer}

\begin{exmp}
Determine whether $2^{70}+3^{70}$ is prime.
\end{exmp}
\begin{solution}
In $\mathbb{Z}_{13}$, powers repeat every $12$, so $2^{70}+3^{70} = 2^{10} + 3^{10} = 10 + 3 = 13$, thus $13 \mid 2^{70}+3^{70}$
\end{solution}

\begin{thm}[Linear Congruence Theorem]
Let $n\in\mathbb{Z}^+$, let $a,b\in\mathbb{Z}$, let $d = gcd(a,n)$. Consider the equation \[ax =b \mod n\]
\begin{enumerate}
\item The equation $ax = b\mod n$ has a solution $x\in\mathbb{Z}$ if and only if $d\mid b$
\item If $x = u$ is a solution (so that $au =b \mod n$), then the general solution is \[x = u + k\frac{n}{d} \text{ for } k\in\mathbb{Z}.\]
\end{enumerate}
\end{thm}

\begin{proof}
This is essentially a restatement of the Linear Congruence Theorem (the LDET) because x is a solution to $ax=b$ mod $n \iff there exist k \in \mathbb{Z} ax=b+kn \iff there exist y \in \mathbb{Z} ax+ny=b$
\end{proof}

\begin{proof} 
\begin{enumerate}
\item TFAE
\begin{enumerate}
\item The equation $ax = b \mod n$ has a solution $x\in \mathbb{Z}$ 
\item Exists $x,y \in \mathbb{Z}$ such that $ax+ny =b$
\item $d\mid b$ (By LDET)
\end{enumerate}
\item Suppose $x=u$ is a solution so that $au = b \mod n$. Thus by the LDET, the general solution to the equation $ax+ny = b$ is \[(x,y) = u + k\frac{n}{d}, ...\] Thus $\displaystyle u + k\frac{n}{d}$ are solutions.
\end{enumerate}
\end{proof}








