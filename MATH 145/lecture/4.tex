\lecture{Sept. 14}


\topic{Class}


\begin{defn}
A class is a collection of sets of the form $$\{x\mid F(x) \text{ is true }\}$$
Where $F(x)$ is a mathematical statement about an unknown set $x$.
\end{defn}

\begin{exmp}
The collection of all sets is the class $\{x\mid x=x\}$
\end{exmp}

\begin{exmp}
If A is a set then $A = \{x\mid x\in A\}$ which is also a class
\end{exmp}


\topic{Mathematical Statement}
\begin{defn}
In the languages of Propositional logic we use symbols from the symbol set
$$\{\neg,\wedge,\vee,\to,\leftrightarrow ,(,)\}$$
together with propositional variable symbols such as $P,Q,R,\dots$

The variable symbols are intended to represent mathematical statements which are either true or false.


In propositional logic, a formula is a non-empty, finite string of symbols (from the above set of symbols) which can be obtained by applying the following rules.

\begin{enumerate}
\item Every propositional variable is a formula.
\item If $F$ is a formula, then so is the string $\neg F$.
\item If $F$ and $G$ are formulas then so is each of the following strings
    \begin{itemize}
        \item $(F\vee G)$
        \item $(F\wedge G)$
        \item $(F\to G)$
        \item $(F\leftrightarrow G)$
    \end{itemize}
\end{enumerate}

A derivation for a formula $F$ is a list of formulas $$F_1,F_2,F_3,\dots$$ with $F=F_k$ for some index $k$ and for each index $l$, either $F_l$ is a propositional variable, or $F_l$ is equal to $F_l=\neg F_i$ for some $i<l$, or $F_l=(F_i \ast F_j)$ for some $i,j<l$ and for some symbol $\ast \in \{\wedge,\vee,\to,\leftrightarrow\}$

\end{defn}

\begin{exmp}
 $$(\neg (\neg P \to Q)\leftrightarrow(R\vee \neg S))$$ is a formula and one possible derivation, with justification on each line, is as follows

\begin{enumerate}
\item $P$
\item $Q$
\item $R$
\item $S$
\item $\neg P$
\item $(\neg P \to Q)$
\item $\neg S$
\item $R\vee \neg S$
\item $\neg (\neg P \to Q)$
\item $(\neg (\neg P \to Q)\leftrightarrow(R\vee \neg S))$
\end{enumerate}

\end{exmp}


\begin{defn}
An \textbf{assignment} of truth-values to the propositional variables is a function $\alpha \colon  \{P,Q,R\dots\} \to \{0,1\}$

For a propositional variable $X$ when $\alpha (X)=1$ we say $X $is true under $\alpha$ and when $\alpha (X)=0$ we say $X$ is false under $\alpha$
\end{defn}

Given an assignment $\alpha \colon \{\text{propositional variables}\} \to {0,1}$
we extend $\alpha$ to a function $\alpha \colon \{\text{formulas}\} \to {0,1}$
by defining $\alpha (F)$ for all formulas $F$ recursively as follows:

When $F=X$ where $X$ is a propositional variable symbol, the value of $\alpha (X)$is already known

When $F=\neg G$ where $G$ is a formula, define $\alpha (F)$ according to the following table
\begin{center}
\begin{tabular}{ c | c }
    $G$ & $\neg G$ \\
    \hline
    1 & 0 \\
    0 & 1 
\end{tabular}
\end{center}


When $F = (G\ast H)$
where G and H are formulas and where $\ast \in \{\wedge,\vee,\to,\leftrightarrow\} $

we define $\alpha (F)$ according to the following table 
\begin{center}
\begin{tabular}{c | c | c | c | c | c }
    $G$ & $H$ & $G\wedge H$ & $G\vee H$ & $G\to H$ & $G\leftrightarrow H$ \\ \hline
    1 & 1   &    1      &     1        &   1      &     1\\
    1 & 0   &    0      &     1       &    0     &      0\\
    0 & 1    &   0      &     1       &    1     &      0\\
    0 & 0    &   0      &     0       &    1    &       1
    
\end{tabular}

\end{center}


