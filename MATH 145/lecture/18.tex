\lecture{Oct. 14}

\begin{exmp}
Given $n,m\in\mathbb{Z}^+$, find \[\sum_{k=1}^n k^m = 1^m+2^m+3^m+\dots\]
\end{exmp}

\begin{solution}
For fixed $n\in \mathbb{Z}^+$, we can find a recursion formula for \[S_m = \sum_{k=1}^n k^m\]

\[S_0 = \sum_{k=1}^n k^0 = n\]
\[S_1 = \sum_{k=1}^n k^1 = \frac{n(n+1)}{2} = \mybinom{n+1}{2}\]

Find \[\sum_{k=1}^n (k+1)^{m+1} - k^{m+1}\] in 2 ways.

\begin{enumerate}
\item
\[\sum_{k=0}^n (k+1)^{m+1} - k^{m+1} = (n+1)^{m+1}\]

\item
\begin{align*}
    &\sum_{k=0}^n (k+1)^{m+1} - k^{m+1} \\
    = & \sum_{k=0}^n ((k^{m+1} + \mybinom{m+1}{1}k^m +  \mybinom{m+1}{2}k^{m-1} + \dots + \mybinom{m+1}{m} k + \mybinom{m+1}{m+1}) - k^{m+1})\\
    =&\mybinom{m+1}{1} \sum_{k=0}^n k^m + \mybinom{m+1}{2} \sum_{k=0}^n k^{m-1} + \dots + \mybinom{m+1}{m} \sum_{k=0}^n k + \mybinom{m+1}{m+1}  \sum_{k=0}^n 1 \\
    \\
    (n+1)^{m+1} = & \mybinom{m+1}{1} \sum_{k=0}^n k^m  + \mybinom{m+1}{2} \sum_{k=0}^n k^{m-1} + \dots + \mybinom{m+1}{m} \sum_{k=0}^n  k + (n+1) \\
\end{align*}

\end{enumerate}

Thus \[S_m = \frac{1}{m+1} ((n+1)^{m+1} - \mybinom{m+1}{2} S_{m-1} - \dots - \mybinom{m+1}{m} S_1 - S_0 - 1)\]
\end{solution}


\begin{thm}
Let $a,b,p,q\in\mathbb{R}$ (or $\mathbb{C}$) with $q\neq 0$ and let $m\in\mathbb{Z}$
Let $(X_n)_{n\geq m}$ be the sequence 
\[x_m = a, x_{m+1} = b, x_n = px_{n-1}+qx_{n-2} \text{ for } n\geq m+2\]

Let $f(x) = x^2 - px - q$ ($f(x)$ is called the characteristic polynomial for the recursion formula)

Suppose that $f(x)$ factors as \[f(x) = (x-u)(x-v)\]
with $u,v\in\mathbb{R}$ (or $\mathbb{C}$) with $u\neq v$

Then there exist $A,B\in \mathbb{R}$ or $\mathbb{C}$ such that 
\[x_n = Au^n + Bv^n\] for all $n\geq m$
\end{thm}

\begin{proof}
    exercise
\end{proof}

\begin{exmp}
Let $(x_n)_{n\geq 0}$ be defined by \[x_0 = 4,x_1 = -1, \ x_n = 3x_{n-1}+10x_{n-2}\]
for $n\geq 2$.

Find a closed form formula for $x_n$
\end{exmp}

\begin{solution}
Let $f(x) = x^2 -3x - 10 = (x-5)(x+2)$.

By the Linear Recursion Theorem, there exists $A,B\in \mathbb{R}$ such that \[x_n = A5^n + B(-2)^n\] for all $n\geq 0.$

To get $x_0 = A5^0+B(-2)^0$, we need
\begin{equation}\label{eq:1}
A+B=4.
\end{equation}

To get $x_1 = A5^1+B(-2)^1$, we need 
\begin{equation}\label{eq:2}
5A-2B = -1.
\end{equation}

Solve \ref{eq:1} and \ref{eq:2} to get \[A=1, \ B = 3\]

Then \[x_n = 5^n + 3(-2)^n\] for all $n\geq 0$.
\end{solution}


\begin{exmp}
There are n points on a circle around a disc. Each of the $\mybinom{n}{2}$ pairs of points is joined by a line segment. Suppose that no three of these line segment have a common point of intersection inside the disc.

Into how many regions is the disc divided by the line segments?
\end{exmp}
\begin{solution}
HINT

Suppose that we have $l$ lines, each of which intersects the circle twice and intersects with the disc in a line segment.

Suppose these $l$ line segments intersect at p points inside the disc. Suppose that no three of these line segment have a common point of intersection inside the disc. Into how many regions is the disc divided by the line segments?
\end{solution}
