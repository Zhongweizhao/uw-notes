\lecture{Oct. 17}
Midterm today 7:00-8:50

RCH 207 Sec 2

RCH 211 Sec 1 A-O

RCH 309 Sec 1 P-Z

\begin{defn}
A \textbf{ring} (with identity) is a set R with two distinct elements $0,1\in R$ and two binary operations $+:R^2\to R$ and $\times :R^2\to R$ where for $a,b\in R$ we write $+(a,b)$ as $a+b$, $\times(a,b)$ as $a\times b$ or $a\cdot b$ or $ab$, such that 
\begin{enumerate}
\item + is associative \[\forall a,b,c\in R\ (a+b)+c = a+(b+c)\]
\item + is commutative \[\forall a,b\in R \ a+b = b+a\]
\item 0 is an identity under + \[\forall a \in R \ a+0=a\]
\item every $a\in R$ has an inverse under + \[\forall a \in R \, \exists b \in R \ a+b = 0\]
\item $\times$ is associative \[\forall a,b,c\in R\ (ab)c = a(bc)\]
\item 1 is an identity under $\times$ \[\forall a\in R \ a\cdot 1 = a \text{ and } 1\cdot a = a \]
\item $\times$ is distributive over + \[\forall a,b,c\in R \ a(b+c) = ab+ac \text{ and } (a+b)c = ac+bc\]
\end{enumerate}

A ring $R$ is \textbf{commutative} when
\begin{enumerate}
\item[8.] $\times$ is commutative \[\forall a,b\in R \ ab=ba\]
\end{enumerate}

A \textbf{field} is commutative ring R such that 
\begin{enumerate}
\item[9.] every nonzero $a\in R $ has an inverse under $\times$. \[\forall 0\neq a\in R \, \exists b\in R \ ab=1\]
\end{enumerate}
\end{defn}

\begin{thm}
$\mathbb{Z}$ is commutative Ring. $\mathbb{Q}$ and $\mathbb{R}$ are fields.
\end{thm}

\begin{exmp}
$\mathbb{N}$ is not a ring (Axiom 4 does not hold)

$\mathbb{Z}$ is not a field (Axiom 9 does not work)
\end{exmp}

\begin{exmp}
The set of \textbf{integers modulo n}, denoted by $\mathbb{Z}_n$, is a ring for $n\in \mathbb{Z}$ with $n\geq 2$. Informally,
\[\mathbb{Z}_n = \{0,1,2,\dots,n-1\}\]
and addition and multiplication modulo n are denoted as follows:

for $a,b\in\mathbb{Z}_n$ \[a+b \in \mathbb{Z}_n \text{ is the remainder when } a+b \in \mathbb{Z} \text{ is divided by }n\]
 \[ab \in \mathbb{Z}_n \text{ is the remainder when } ab \in \mathbb{Z} \text{ is divided by }n\]
 
In $\mathbb{Z}_6$:

\begin{center}
\begin{tabular}{l|l|l|l|l|l|l}
 + & 0 & 1 & 2 & 3 & 4 & 5 \\ \hline
 0 & 0 & 1 & 2 & 3 & 4 & 5 \\\hline
 1 & 1 & 2 & 3 & 4 & 5 & 0 \\\hline
 2 & 2 & 3 & 4 & 5 & 0 & 1 \\\hline
 3 & 3 & 4 & 5 & 0 & 1 & 2 \\\hline
 4 & 4 & 5 & 0 & 1 & 2 & 3 \\\hline
 5 & 5 & 0 & 1 & 2 & 3 & 4
\end{tabular}

\begin{tabular}{l|l|l|l|l|l|l}
 $\times$ & 0 & 1 & 2 & 3 & 4 & 5 \\\hline
 0 & 0 & 0 & 0 & 0 & 0 & 0 \\\hline
 1 & 0 & 1 & 2 & 3 & 4 & 5 \\\hline
 2 & 0 & 2 & 4 & 0 & 2 & 4 \\\hline
 3 & 0 & 3 & 0 & 3 & 0 & 3 \\\hline
 4 & 0 & 4 & 2 & 0 & 4 & 2 \\\hline
 5 & 0 & 5 & 4 & 3 & 2 & 1 \\
\end{tabular}

\end{center}
We shall see that $\mathbb{Z}_n$ is a field if and only if $n$ is prime
\end{exmp}

\begin{exmp}
The field of \textbf{complex numbers} is the set \[\mathbb{C} = \mathbb{R}^2 = \{(x,y)\mid x\in \mathbb{R},y\in\mathbb{R}\}\]
and for $x,y\in\mathbb{R}$ we write \[0 = (0,0), 1 = (1,0), i = (0,1), x = (x,0), iy = yi = (0,y), x+iy = (x,y)\]
and we define $+$ and $\times$ as follows

for $a,b,c,d\in R$ \[(a+ib)+(c+id) = (a,b)+(c,d) = (a+c,b+d) = (a+c) + i(b+d)\]
\[(a+ib)(c+id) = (ac-bd) + i(ad+bc)\]
\end{exmp}

\begin{rem}
Check that when $(a,b)\neq (0,0)$, $a+ib$ has an inverse.
\end{rem}

\begin{exmp}
\[\mathbb{Z}[\sqrt{2}] = \{a+b\sqrt{2} \mid a,b\in \mathbb{Z}\} \subseteq \mathbb{R}\] is a ring.
\[\mathbb{Q}[\sqrt{2}] = \{a+b\sqrt{2} \mid a,b\in \mathbb{Q}\} \subseteq \mathbb{R}\] is a field.
\[\mathbb{Z}[\sqrt{3}i] = \{a+b\sqrt{3}i \mid a,b\in \mathbb{Z}\} \subseteq \mathbb{C}\] is a ring.
\[\mathbb{Q}[\sqrt{3}i] = \{a+b\sqrt{3}i \mid a,b\in \mathbb{Q}\} \subseteq \mathbb{C}\] is a field.
\end{exmp}

\begin{exmp}
If $R$ is a ring (usually commutative), the set of polynomials \[f(x) = c_0 + c_1x + c_2x^2+ \dots + c_nx^n\] with coefficients $c_k\in R$ is a ring (under addition and multiplication of polynomials) which we denote by $R[x]$
\end{exmp}

\begin{exmp}
If $R$ is a ring, the set of all $n\times n$ matrices
\begin{math}
 \begin{bmatrix}
 a_{11} & a_{12} & \cdots & a_{1n} \\
 a_{21} & a_{22} & \cdots & a_{2n} \\
 \vdots & \vdots & \ddots & \vdots \\
 a_{n1} & a_{n2} & \cdots & a_{nn}
 \end{bmatrix}
 \end{math}
 with entries $A_{kl} = a_{kl} \in R$ is a ring, which we denote by $M_n(R)$ (or $M_{n\times n} (R)$) (under addition and multiplication of matrices.)
\end{exmp}

\begin{rem}
Matrices
\end{rem}